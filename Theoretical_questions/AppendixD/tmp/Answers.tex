\documentclass[lang=cn,a4paper,newtx,bibend=bibtex]{elegantpaper}
\usepackage{env}
\title{Problems of Appendix-D}
\author{张志心 \ 计科2106}
\date{\zhdate{2024/05/28}}
\pgfplotsset{compat=1.17}
\addbibresource[location=local]{reference.bib}
\begin{document}
\maketitle

\begin{prob}[D.5]
  Let ${\cal X}$ be the set of all bounded and unbounded sequences
  of complex numbers. Show that the function $d$ given by
  \begin{equation*}
    \forall x=(\xi_j),\ \forall y=(\eta_j),\ \ 
    d(x,y) = \sum_{j=1}^{\infty} \frac{1}{2^j}
    \frac{|\xi_j-\eta_j|}{1+|\xi_j-\eta_j|}
  \end{equation*}
  is a metric on ${\cal X}$.
\end{prob}

\begin{proof}~~\\根据 Def D.3:
\begin{enumerate}
\item 非负性:\(\forall x=(\xi_j),\ \forall y=(\eta_j),\ \ 
    d(x,y) = \sum_{j=1}^{\infty} \frac{1}{2^j}
    \frac{|\xi_j-\eta_j|}{1+|\xi_j-\eta_j|}\geq 0\);
    \item 不可分与同一性:若 \(x = y\),即 \(\sum_{j=1}^{\infty} \frac{1}{2^j}\frac{|\xi_j - \eta_j|}{1+|\xi_j - \eta_j} = 0\),反之,若  \(\sum_{j=1}^{\infty} \frac{1}{2^j}\frac{|\xi_j - \eta_j|}{1+|\xi_j - \eta_j} = 0\),因为 \(\forall j, \frac{1}{2^j}\frac{|\xi_j - \eta_j|}{1+|\xi_j - \eta_j} \geq 0\),所以 \(\forall j, \frac{1}{2^j}\frac{|\xi_j - \eta_j|}{1+|\xi_j - \eta_j} = 0\Rightarrow \forall j, \xi_j = \eta_j \Rightarrow x = y\) ;
    \item  对称性:\(\forall x=(\xi_j),\ \forall y=(\eta_j),\ \ 
    d(x,y) = \sum_{j=1}^{\infty} \frac{1}{2^j}
    \frac{|\xi_j-\eta_j|}{1+|\xi_j-\eta_j|} = \sum_{j=1}^{\infty} \frac{1}{2^j}
    \frac{|\eta_j-\xi_j|}{1+|\eta_j-\xi_j|} = d(y,x)\);
    \item 三角不等式:因为 $f(t) = \frac{t}{1+t}$ 单调递增,所以由 $|a+b| \leq |a|+|b|$ 可以得到 $f(|a+b|) \leq f(|a|+|b|)$。所以有
\[\frac{|a+b|}{1+|a+b|} \leq \frac{|a|+|b|}{1+|a|+|b|}\leq \frac{|a|}{1+|a|+|b|} + \frac{|b|}{1+|a|+|b|} \leq \frac{|a|}{1+|a|} + \frac{|b|}{1+|b|},\]
对于 $\forall x = (x_i)_j, y=(\eta_j), z=(\zeta_j)$,取 $a = \xi_j - \eta_j, b = \eta_j - \zeta_j$,有
\[d(x,z) = \sum_{j=1}^{\infty} \frac{1}{2^j} \frac{|\xi_j - \zeta_j|}{1+|\xi_j - \zeta_j|}\leq \sum_{j=1}^{\infty} \frac{1}{2^j} \left[\frac{|\xi_j - \eta_j|}{1+|\xi_j-\eta_j|} + \frac{|\eta_j - \zeta_j|}{1+|\eta_j-\zeta_j|}\right] = d(x, y) + d(y, z).\]
\end{enumerate}
根据 (1,2,3,4) 可以得到 $d$ 是 $\mathcal{X}$ 上对度量。
\end{proof}

\begin{prob}[D.16]
  The completeness depends on the metric.
  For the metric 
  $d_1(x,y)=\int_a^b|x(t)-y(t)|\,\dif t$,
  show that the metric space $({\cal C}[a,b],d_1)$
  is not complete.
\end{prob}

\begin{proof}
    取
    \begin{equation*}
        f_n(t) = 
        \begin{aligned}
            \left\{
                \begin{array}{ll}
                    -1, & a\leq t\leq \dfrac{a+b}2 - \dfrac{b-a}{2n}, \\
                    \dfrac{2n}{b-a}\left(t - \dfrac{a+b}2\right), & \dfrac{a+b}2 - \dfrac{b-a}{2n} < t < \dfrac{a+b}2 + \dfrac{b-a}{2n}, \\
                    1. & \dfrac{a+b}2 + \dfrac{b-a}{2n} \\
                \end{array}
            \right.
        \end{aligned}
    \end{equation*}
    对任意 $m>n$,有 $d_1(f_n, f_m) = \dfrac{b-a}2(\dfrac 1n - \dfrac 1m) \rightarrow 0 ~(m,n\rightarrow \infty$,因此 $\{f_n\}$ 是 $({\cal C}[a,b],d_1)$ 上的 Cauchy 列。
    假设 $\{f_n\}$ 收敛于 $f\in \cal C$,则任意 $a<c<b$,有
    \begin{equation*}
        \lim_{n\rightarrow \infty} \int_a^b |f_n(x) - f(x)| \dd x
        = \lim_{n\rightarrow \infty} \int_a^c |f_n(x) - f(x)| \dd x
        + \lim_{n\rightarrow \infty} \int_c^b |f_n(x) - f(x)| \dd x
        = 0.
    \end{equation*}
    即
    \begin{equation*}
        \lim_{n\rightarrow \infty} \int_a^c |f_n(x) - f(x)| \dd x
        = \lim_{n\rightarrow \infty} \int_c^b |f_n(x) - f(x)| \dd x
        = 0.
    \end{equation*}
    因为对任意的 $a<c<\frac {a+b}2$,都存在 $N\in \mathbb{N}^+$ 使得 $\forall n>N, \forall a<x<c, f_n(x) = -1$,所以 $\{f_n\}$ 在 $[a,c]$ 上收敛于常值 $-1$。由 $c$ 的任意性可得 $\{f_n\}$ 在 $[a,\frac{a+b}2)$ 上收敛于 $-1$。同理可知 $\{f_n\}$ 在 $(\frac {a+b}2,b]$ 上收敛于 $1$。即 $f$ 在 $[a,\frac{a+b}2)$ 取值为 $-1$,在 $(\frac {a+b}2,b]$ 上取值为 $1$。无论 $f$ 在 $\frac {a+b}2$ 上如何取值都不可能连续,因此 $\{f_n\}$ 在 $\cal C$ 上不收敛。即 $(\cal C, d_1)$ 不完备。
\end{proof}

\begin{prob}[D.40]
  Show that the $\ell^p$ space in Definition D.39
  is complete for $p\ge 1$.
\end{prob}

\begin{proof}
设 $(x_n)$ 是 $(\ell^p, d)$ 上的 Cauchy 列 \(x_n = (\xi_{n,j})_{j\geq 1}\),即
\[\lim_{n,m\to \infty} d(x_n, d_m) = \lim_{n, m\to \infty} \left(\sum_{j=1}^{\infty} |\xi_{n, j} - \xi_{m, j}|^p\right)^{\frac1{p}} = 0,\]
所以,$\forall \epsilon > 0, \exists N, \forall m, n > N$,有 $d(x_n, d_m) < \epsilon$,即 $|\xi_{n, j} - \xi_{m, j}| < d(x_n, d_m) < \epsilon$,所以 $\forall j, (\xi_{\cdot, j})$ 是 $\mathbb{C}$ 上的 Cauchy 列,所以 $\exists \xi_j \in \mathbb{C}$ 满足 $\xi_j = \lim_{n\to \infty} \xi_{n, j}$。令 $x = (\xi_j)_{j\geq 1}$。对于每一个固定的 $i$ 和上述 $m,n$,有 $\sum_{j=1}^i |\xi_{n, j} - \xi_{m,j}|^p < \epsilon^p$,令 $m\to \infty$,有 $\sum_{j=1}^i |\xi_{n, j} - \xi_j|^p \leq \epsilon^p$,再令 $i\to \infty$,有 $\sum_{j=1}^{\infty} |\xi_{n,j} - \xi_j|^p \leq \epsilon^p$,所以 $x_n - x \in \ell^p$。因此 $\lim_{n\to\infty} x_n = x$ ,根据三角不等式
\[
\left(\sum_{j=1}^{\infty} |\xi_j|^p\right)^{\frac1{p}} = d(x_n - x, x) \leq d(x_n, 0) + d(x_n - x, 0) < \infty.
\]
所以 $x\in \ell^p$,因此 $(\ell^p, d), p\geq 1$ 是完备的。
\end{proof}

\begin{prob}[D.42]
  Show that the sequence space $c_{00}$
  in Notation 24
  is a dense subset of $\ell^p$
  in Definition D.39 with $p\in[1,\infty)$.
\end{prob}

\begin{proof}
$\forall (a_n)_{n\geq 1}\in \ell^p$,有 $\sum_{n\geq 1} |a_n|^p < \infty$,所以 $\forall \epsilon > 0, \exists N \in \mathbb{N}\quad \text{s.t.}\quad \sum_{n\geq N} |a_n|^p < \epsilon^p$,构造 $(b_n) \in c_{00}$ 为 $b_n = \Cases{a_n, n< N\\ 0, \text{Others}}$。所以 $d((a_n), (b_n)) = \left(\sum_{n\geq N} |a_n|^p\right)^{\frac1{p}} < \epsilon$,根据 Def D.17,$c_{00}$ 是 $\ell^p$ 的稠密子集。
\end{proof}

\begin{prob}[D.49]
  Show that $\emptyset$ and ${\cal X}$
  are both open and closed.
\end{prob}

\begin{proof}
因为 $\mathcal{X} = \mathcal{X} \fxg \emptyset, \emptyset = \mathcal{X} \fxg \mathcal{X}$,所以只需要证明 $\emptyset, \mathcal{X}$  都是开集。因为 $\forall x \in \mathcal{X}, x\notin \emptyset$,所以 $\emptyset$ 为开集(虚真命题);因为 $\forall x \in \mathcal{X}, \forall r > 0, B_r(x) \in \mathcal{X}$,所以 $\mathcal{X}$ 是开集。
\end{proof}

\begin{prob}[D.52]
  If a closed set $F$ in a metric space ${\cal X}$
  does not contain any open set,
  then ${\cal X}\setminus F$ is dense in ${\cal X}$.
\end{prob}

\begin{proof}
反证法,若 $\mathcal{X}\setminus F$ 在 $\mathcal{X}$ 中不稠密,那么 $\exists x \in \mathcal{X}, \epsilon > 0, B_r(x) \cap \mathcal{X} \setminus F = \emptyset$,所以 $B_r(x) \subset F$,所以 $\emptyset \neq E:= \{y : d(x, y) < \frac{r}2\} \subset F$ 是一个开集(因为 $\forall y \in E, \exists r_1 = \frac{\frac{r}2 - d(x, y)}2 > 0, \forall z \in B_{r_1}(y), d(x,z) \leq d(x, y) + d(y, z) \leq r_1 + d(x, y) < \frac{r}2, \Rightarrow z \in E, \Rightarrow B_{r_1} \subset E $),与 $F$ 中不包含任何开集矛盾。\\所以 $\mathcal{X} \setminus F$ 在 $\mathcal{X}$ 中稠密。
\end{proof}

\begin{prob}[D.112]
  What is the connection between the logical statements in (D.27) and (D.30)?
\end{prob}

\begin{solution}
    (D.30) 是 (D.27) 的充分不必要条件。
\\
    假设 (D.30) 成立,对任意的 $\epsilon > 0$,取 $\delta = \dfrac{\epsilon}L$,则
    \begin{equation*}
        \forall x,y\in \mathcal{X}, d_{\mathcal{X}}(x,y)<\delta \Rightarrow d_{\mathcal{Y}}(x,y)<L\delta = \epsilon.
    \end{equation*}
    因此 (D.27) 成立。
\\
    反之,若 (D.27) 成立,取 $f(x) = \sqrt x, \mathcal{X} = [0,1], \mathcal{Y} = [0,1]$。
    对任意 $\epsilon > 0$,取 $\delta = \epsilon^2$,则
    \begin{equation*}
        \forall x,y\in [0,1], |x-y|<\epsilon^2 \Rightarrow |\sqrt x - \sqrt y| = \left|\dfrac{x-y}{\sqrt x + \sqrt y}\right| < \dfrac{\epsilon^2}{\epsilon} = \epsilon.
    \end{equation*}
    因此 (D.27) 成立。但因为 $f(x)$ 的导数无界,所以 (D.30) 不成立。
\end{solution}

\begin{prob}[D.137]
  Show that the \emph{Hilbert cube}
  in the metric space $\ell^2$, % $(\ell^2, \|\cdot\|_2)$, 
  \begin{equation*}
    C := \left\{
      (x_n)_{n\in \mathbb{N}^+}:\ x_n\in \left[0,\frac{1}{n}\right]
      \right\}, 
  \end{equation*}
  is sequentially compact.
\end{prob}

\begin{proof}
    $C$ 显然是闭集。因此根据 Lem D.136,只需证明 $C$ 完全有界。
\\
    对任意 $\epsilon>0$,存在 $n\in \mathbb{N}^+$ 使得 $\sum_{k=n+1}^{\infty}\dfrac 1{k^2} < \dfrac{\epsilon^2}2$。令
    \begin{equation*}
        N_\epsilon = \left\{(x_1, x_2, \dots, x_n, 0, 0, \dots)~:~ \forall k = 1, \dots, n, x_k = \dfrac{j_k\epsilon}{\sqrt{2n}}, j_k = 0, 1, \dots, \left\lfloor \dfrac{\sqrt{2n}}{k\epsilon}\right\rfloor\right\},       
    \end{equation*}
    则对任意 $y = (y_n)_{n\in \mathbb{N^+}}$,存在 $x\in N_\epsilon$,使得
    \begin{equation*}
        d(x,y)^2 = \sum_{k=1}^\infty (x_k-y_k)^2 = \sum_{k=1}^n (x_k-y_k)^2 + \sum_{k=n+1}^{\infty} y_k^2 < \sum_{k=1}^n \left(\dfrac{\epsilon}{\sqrt{2n}}\right)^2 + \sum_{k=n+1}^{\infty} \left(\dfrac 1k\right)^2 < \dfrac{\epsilon^2}2 + \dfrac{\epsilon^2}2 = \epsilon^2.
    \end{equation*}
    故 $C$ 完全有界。
    因此 $C$ 列紧。
\end{proof}

\begin{prob}[D.159]
  Denote by $\Omega\subset \mathbb{R}^n$ a bounded open convex set.
  For $M_1,M_2\in\mathbb{R}^+$,
  show that the set
  \begin{equation*}
    {\cal F} := \left\{f\in {\cal C}^{(1)}(\overline{\Omega}) :
      \forall x\in \Omega,\ |f(x)|\le M_1;\, 
      \|\nabla f(x)\|\le M_2
    \right\}
  \end{equation*}
  is sequentially compact.
\end{prob}

\begin{proof}~~\\
    \textbf{一致有界性}:因为 $\forall f\in \mathcal{F}, \forall x\in \Omega, |f(x)|\leq M_1$,所以 $\mathcal{F}$ 是一致有界的。\\
    \textbf{等度连续性}:对任意 $f\in \mathcal{F}, x, y \in \Omega$,因为 $\Omega$ 是凸集,所以 线段 $[x,y]\subset \Omega$。
    令 $\phi(t) = f((1-t)x + ty)$,则 $\phi(0) = f(x), \phi(1) = f(y), \phi'(t) = (y-x)^T\nabla f((1-t)x + ty)$。根据微分中值定理有
    \begin{equation*}
        |f(y)-f(x)| = |\phi(1)-\phi(0)| = |\phi'(\xi)| = |(y-x)^T\nabla f((1-\xi)x + \xi y| \leq |y-x||\nabla f((1-\xi)x + \xi y| \leq M_2|y-x|.
    \end{equation*}
    因此 $f\in \mathcal{F}$ Lipschitz 连续且有一致的 Lipschitz 常数 $M_2$。对任意 $\epsilon>o$,取 $\delta = \dfrac{\epsilon}{M_2}$,则对任意 $f\in \mathcal{F}, x,y\in \Omega, |x-y|<\delta$,有 $|f(x)-f(y)|<M_2\delta=\epsilon$。
\\
    由 Ascoli-Arzela 定理(Theorem D.158),$\mathcal{F}$ 是列紧函数空间。
\end{proof}

\begin{prob}[D.172]
  If the radius of convergence of $f$ in
  Example D.171
  is $+\infty$,
  does $(T_n)_{n\in\mathbb{N}^+}$ converge to $f$
  uniformly or locally uniformly?
\end{prob}

\begin{solution}
    局部一致收敛。由 Example D.170,对任意 $r>0$,$\{T_n(x)\}$ 在闭区间 $[a-r,a+r]$ 上局部一致收敛于 $f(x)$。下面举反例证明 $\{T_n(x)\}$ 在 $\mathbb{R}$ 上不一致收敛于 $f$。\\
    考虑 $f(x) = e^x, a = 0$。因为 $f(x)$ 在 $\mathbb{R}$ 上解析,所以其泰勒级数 $T_n(x) = \sum_{k=0}^n \dfrac{x^n}{n!}$ 的收敛半径为 $+\infty$。\\
    取 $\epsilon=1$,对任意的 $n\in \mathbb{N}^+$,取 $x=n+1$,则
    \begin{equation*}
        |T_n(x)-f(x)| = |\sum_{k=0}^n \dfrac{x^k}{k!} - e^x| = \sum_{k=n+1}^n \dfrac{(n+1)^k}{k!} > \dfrac{(n+1)^{n+1}}{(n+1)!} > 1.
    \end{equation*}
    所以 $\{T_n(x)\}$ 在 $\mathbb{R}$ 上不一致收敛于 $f(x)$。
\end{solution}

\end{document}