%!TEX program = xelatex
% 完整编译: xelatex -> biber/bibtex -> xelatex -> xelatex
\documentclass[lang=cn,a4paper,newtx,bibend=bibtex]{elegantpaper}
\usepackage{tikz}
\usepackage{pgfplots}

\title{样条设计文档}
\author{张志心 \ 混合2106}

\date{\zhdate{2024/01/04}}

% \qedhere to make the square straight after

\usepackage{array}
\usepackage{tcolorbox}
\usepackage{tikz}
\usepackage{pgfplots}
\usepackage{float}
\usepackage{bm}
\usepackage{amsmath}
\usepackage{graphicx}
\usepackage{subcaption}

\newtcolorbox{prob}[1][]{
  colframe=gray,
  colback=white,
  boxrule=1.5pt, % 控制外边框线的宽度
  sharp corners, % 使用直角边框
  fonttitle=\bfseries,
  title=#1
}

\newcommand{\ccr}[1]{\makecell{{\color{#1}\rule{1cm}{1cm}}}}
\newcommand{\xB}{\bm{x}}
\newcommand{\XB}{\bm{X}}
\newcommand{\yB}{\bm{y}}
\newcommand{\gB}{\bm{g}}
\newcommand{\uB}{\bm{u}}
\newcommand{\vB}{\bm{v}}
\newcommand{\wB}{\bm{w}}
\newcommand{\wanwan}[1]{\tilde{#1}}
\newcommand{\dd}{\mathrm{d}}
\newcommand{\RBB}{\mathbb{R}}
\newcommand{\CBB}{\mathbb{C}}
\newcommand{\FBB}{\mathbb{F}}
\newcommand{\SBB}{\mathbb{S}}
\newcommand{\FM}{\mathcal{F}}
\newcommand{\SM}{\mathcal{S}}
\newcommand{\LM}{\mathcal{L}}
\newcommand{\VM}{\mathcal{V}}
\newcommand{\CM}{\mathcal{C}}
\newcommand{\apart}[3]{\frac{\partial^{#3}{#1}}{\partial {#2}^{#3}}}
\newcommand{\dpart}[3]{\dfrac{\partial^{#3}{#1}}{\partial {#2}^{#3}}}
\newcommand{\upset}[2]{\stackrel{#1}{#2}}
\newcommand{\domf}{\textrm{dom}\;f}
\newcommand{\Int}[4]{\int_{#1}^{#2}{#3}{\dd {#4}}}
\newcommand{\indot}[2]{\langle {#1}, {#2} \rangle}
\newcommand{\functiontype}[3]{\FM_{#1}^{#2,#3}(\RBB^n)}
\newcommand{\normgen}[1]{\left\| #1 \right\|}
\newcommand{\strongconvextype}[2]{\SM_{#1}^{#2}(\RBB^n)}
\newcommand{\argmin}{\mathop{\rm argmin}}
\newcommand{\LII}{\lstinline[language=C++]}

\addbibresource[location=local]{reference.bib}

\begin{document}

\maketitle

\section{Spline 基类}

\begin{lstlisting}[language = C++]
    using NUM = double;
    template <class Type = NUM>
    class Spline;
\end{lstlisting}

包含成员如下:

\begin{enumerate}
\item \lstinline[language=C++]{int N,Size}:样条结点数为 Size,坐标从 0 到 $N$;
\item \lstinline[language=C++]{std::<vector> X, Y}:输入的样条结点 $(X_i, Y_i)$;
\item \lstinline[language=C++]{bool _spline_builded}:表示样条是否建立完成;
\item \lstinline[language=C++]{std::vector<Poly_n<Type>> splines}:样条,分段多项式形式;
\item \lstinline[language=C++]{bool _self_checked}:样条输入是否自检完成;
\item \lstinline[language=C++]{void __self_checked__()}:样条自检函数;
\item \lstinline[language=C++]{Type operator(const Type &x)}:虚函数,样条求值函数;
\item \lstinline[language=C++]{Type get_error_i (const Function<Type>& func, vector<Type>& X) const}:对于点列 $X$,原函数 $func$ 计算无穷范数;
\item \lstinline[language=C++]{Type get_error_1 (const Function<Type>& func, vector<Type>& X) const}:1范数;
\item \lstinline[language=C++]{Type get_error_2 (const Function<Type>& func, vector<Type>& X) const}:2范数;
\item \lstinline[language=C++]{Type get_midpoint_error (const Function<Type>& func) const}:对于插值区间中点处的误差向量求无穷范数;
\item \lstinline[language=C++]{virtual const std::string to_python ()}:虚函数,将样条转化为 python 格式(以分段多项式的形式输出)。
\end{enumerate}

自检函数和构造函数:

\begin{lstlisting}[language = C++]
    Spline (const vector<Type> &x, const vector<Type> &y) : 
        X(x.size()), Y(y.size()), Size(x.size()), N(x.size() - 1), 
		_spline_builded(0), splines(x.size() - 1), _self_checked(0) {
			custom_assert(x.size() == y.size(), "Spline: Input X, Y have different length.");
			custom_assert(x.size() >= 2, "Spline: Input shall not be too short.");
        	X = x; Y = y;
    }
    bool _self_checked{};
    void __self_check__() {
        custom_assert(is_sorted(X.begin(), X.end()), "Spline: Input points not be sorted by X-coordinates.");
        custom_assert(X.end()==unique(X.begin(), X.end()), "Spline: Input points X-coordinates duplicate.");
    }
\end{lstlisting}

\section{ppForm Spline 类}

\begin{lstlisting}[language = C++]
    template<class Type, int Order>
    class ppForm_Spline : public Spline<Type> 
\end{lstlisting}

包含成员如下:
\begin{enumerate}
\item \LII{void buildLinearSpline()}:建立线性样条;
\item \LII{void buildCubicSpline()}:建立三次样条;
\item \LII{void build()}:建立样条接口;
\end{enumerate}

构造方法为直接调用基类构造:

\begin{lstlisting}[language = C++]
    ppForm_Spline(const vector<Type> &x, const vector<Type> &y) : 
    Spline<Type>(x, y) {}
\end{lstlisting}

求值函数,找到所在的分段多项式,调用多项式求点值:

\begin{lstlisting}[language = C++]
    Type operator() (const Type &x) const {
        // find which interval x lies in.
        custom_assert(this->_spline_builded, "ppForm Spline : Spline have not been built yet.");
        custom_assert(x >= X[0]-1e-10 && x <= X.back()+1e-10, "ppForm Spline : Input x is out of range.");
        auto getValue = [&](int id, const Type& x) -> Type {
            return splines[id](x - X[id]);
        };
		if(x >= X[N]) return Y[N];
        return getValue(upper_bound(X.begin(), X.end(), x) - X.begin() - 1, x);
    }
\end{lstlisting}

样条求解部分(目前只实现了1次和3次):

\begin{lstlisting}[language = C++]
    /******************* begin Linear Spline *******************/
    void buildLinearSpline() { // build straightly 
        for(int i = 0; i < N; ++i) {
            splines[i] = Poly_n<Type>(vector<Type>({Y[i], (Y[i+1]-Y[i])/(X[i+1]-X[i])}));
        }
    }
    /******************* end Linear Spline *******************/

    /******************* begin Cubic Spline *******************/
    void buildCubicSpline(Cubic_Spline_Condition<Type, Nature> cond, Vec<Type> &x, 
	Vec<Type> &y, Vec<Type> &z, Vec<Type> &b, const vector<vector<Type>>& dd,
	const vector<Type> &lambda, const vector<Type> &mu) {
        // 构造三对角矩阵和右端项
        b[0] = 0; b[N] = 0; y[0] = y[N] = 1;
		for(int i = 1; i < N; ++i) b[i] = 6*dd[1][i];

		for(int i = 1; i < N; ++i) y[i] = 2;
		for(int i = 1; i < N; ++i) x[i] = lambda[i]; x[0] = 0;
		for(int i = 1; i < N; ++i) z[i-1] = mu[i]; z[N-1] = 0;
    }
    void buildCubicSpline(Cubic_Spline_Condition<Type, Complete> cond, Vec<Type> &x, 
	Vec<Type> &y, Vec<Type> &z, Vec<Type> &b, const vector<vector<Type>>& dd,
	const vector<Type> &lambda, const vector<Type> &mu) {
        // 构造三对角矩阵和右端项
        b[0] = 6*(cond.sa - dd[0][0]) / (X[0] - X[1]);
		b[N] = 6*(cond.sb - dd[0][N-1]) / (X[N] - X[N - 1]);
		for(int i = 1; i < N; ++i) b[i] = 6*dd[1][i];

		for(int i = 0; i <= N; ++i) y[i] = 2;
		for(int i = 1; i < N; ++i) x[i] = lambda[i]; x[0] = 1;
		for(int i = 1; i < N; ++i) z[i-1] = mu[i]; z[N-1] = 1;
    }
    void buildCubicSpline(Cubic_Spline_Condition<Type, Second_Derivatives> cond, Vec<Type> &x, 
	Vec<Type> &y, Vec<Type> &z, Vec<Type> &b, const vector<vector<Type>>& dd,
	const vector<Type> &lambda, const vector<Type> &mu) {
        // 构造三对角矩阵和右端项
        b[0] = cond.sa; b[N] = cond.sb; y[0] = y[N] = 1;
		for(int i = 1; i < N; ++i) b[i] = 6*dd[1][i];

		for(int i = 1; i < N; ++i) y[i] = 2;
		for(int i = 1; i < N; ++i) x[i] = lambda[i]; x[0] = 0;
		for(int i = 1; i < N; ++i) z[i-1] = mu[i]; z[N-1] = 0;
    }
    /******************* end Cubic Spline *******************/

    void build() {
        if(this->_spline_builded) { return; }
        if(!this->_self_checked) { this->__self_check__(); this->_self_checked = 1; }
        if(Order == 1) buildLinearSpline();
        else { std::cerr << "ppForm_Spline: build() not implemented.\n"; return; }
        this->_spline_builded = 1;
    }

    template <Cubic_Spline_Type cType>
    void build(Cubic_Spline_Condition<Type, cType> cond) {
        if(this->_spline_builded) { return; }
        if(!this->_self_checked) { this->__self_check__(); this->_self_checked = 1; }
        if(Order == 3) {
			// divided differences calculate
			vector<vector<Type>> dd(2);
			dd[0].resize(N + 1); dd[1].resize(N + 1);
			for(int i = 0; i < N; ++i) dd[0][i] = (Y[i] - Y[i+1]) / (X[i] - X[i+1]);
			for(int i = 1; i < N; ++i) dd[1][i] = (dd[0][i] - dd[0][i-1]) / (X[i+1] - X[i-1]);

			// lambda & mu
			vector<Type> lambda(N), mu(N);
			for(int i = 1; i < N; ++i) lambda[i] = (X[i] - X[i-1]) / (X[i + 1] - X[i - 1]),
										   mu[i] = (X[i+1] - X[i]) / (X[i + 1] - X[i - 1]);

			
			// construct linear system Ax = b
			Vec<Type> x(N), y(N+1), z(N), b(N+1);
			buildCubicSpline(cond, x, y, z, b, dd, lambda, mu);
			Vec<Type> M = Thomas(y, z, x, b);
			Vec<Type> m(M.Size());

			// calculate m
			for(int i = 0; i < N; ++i) m[i] = dd[0][i]-(1.0/6)*(M[i+1]+2*M[i])*(X[i+1]-X[i]);
			m[N] = dd[0][N-1]-(1.0/6)*(M[N-1]+2*M[N])*(X[N-1]-X[N]);

			// get ppForm
			splines[0] = Poly_n<Type>(vector<Type>({Y[0], m[0], M[0]/2.0, (M[1]-M[0])/(X[1]-X[0])/6.0}));
			for(int i = 1; i < N; ++i) {
            	splines[i] = Poly_n<Type>(vector<Type>({Y[i], m[i], M[i]/2.0, (M[i+1]-M[i])/(X[i+1]-X[i])/6.0}));
        	}
		}
        else { std::cerr << "ppForm_Spline: build() not implemented.\n"; return; }
        this->_spline_builded = 1;
    }

\end{lstlisting}

\section{BForm Spline 类}

\begin{lstlisting}[language=C++]
    template <class Type, int Order>
    class B_Spline : public Spline<Type>
\end{lstlisting}

包含成员如下:
\begin{enumerate}
\item \LII{vector<Type> a, t}:B样条系数,B 样条基函数系数和延拓后的所有点(共 $N+Order+Order+1$ 个)。
\item \LII{int offset}:真实坐标为 $[-offset, N]$;
\item \LII{bool _calculate_ppForm}:是否已经将样条转化为分段多项式形式保存在 \LII{splines} 中;
\item \LII{bool _B_Spline_Base_build}:是否已经建立完成所有B样条基函数;
\item \LII{Type T(int i) const}:返回真实的 $t_i, i = -offset+1, \cdots, N$;
\item \LII{class B_Spline_Base}:B样条基函数类,支持加,乘多项式,数乘,求高阶导等运算,本质是维护相关分段多项式;
\item \LII{vector<vector<B_Spline_Base>>}:B 样条基函数。
\item \LII{void build_B_Spline()}:递推建立样条基函数;
\item \LII{const B_Spline_Base get_B_Spline(int I, int n) const}:返回真实下标的 $B_{t_I}^n$;
\item \LII{void build()}:建立 B 样条函数。
\end{enumerate}

构造函数:

\begin{lstlisting}[language=C++]
    B_Spline (const vector<Type> &x, const vector<Type> &y) : Spline<Type>(x, y), _calculate_ppForm(0), _B_Spline_Base_build(0) {
		this->__self_check__();
		this->_self_checked = 1;
		Type d = 0; // calculate difference
		for(int i = 0; i < N; ++i) d += X[i+1] - X[i];
		d /= N;
		offset = Order;
		for(int i = offset; i; --i) t.push_back(X[0] - i*d);
		for(int i = 0; i <= N; ++i) t.push_back(X[i]);
		for(int i = 1; i <= offset; ++i) t.push_back(X[N] + i*d);
		build_B_Spline();
	}
\end{lstlisting}

求值函数,找到支撑集包含该点的所有基函数,求值并相加:

\begin{lstlisting}[language=C++]
    Type operator() (const Type &x) const {
		custom_assert(this->_spline_builded, "B Spline : Spline have not been built yet.");
        custom_assert(x >= X[0] - 1e-10 && x <= X.back() + 1e-10, "B Spline : Input x is out of range.");
		if(x >= X.back()) return Y.back();
		if(x <= X[0]) return Y[0];
		Type ret = 0;
		int pos = upper_bound(t.begin(), t.end(), x) - t.begin() - 1 - offset;
		for(int i = max(0, pos); i <= pos + Order && i < N + Order; ++i)
			ret += a[i] * get_B_Spline(i-offset+1, Order)(pos, x);
		return ret;
	}
\end{lstlisting}

B 样条基函数的递推计算:

\begin{lstlisting}[language=C++]
    void build_B_Spline() {
		B_Spline_Base_store.resize(N+offset+offset+1);
		for(int i = -offset+1; i <= N+offset; ++i) {
			B_Spline_Base_store[i+offset].resize(Order+1);
			B_Spline_Base_store[i+offset][0] = B_Spline_Base(i);
		}
		for(int j = 1; j <= Order; ++j) 
		for(int i = -offset+1; i <= N+offset-j; ++i) {
			B_Spline_Base_store[i+offset][j] = 
				B_Spline_Base_store[i+offset][j-1] * (Poly_n<Type>(vector<Type>{-T(i-1), 1}) / (T(i+j-1)-T(i-1)))
			  + B_Spline_Base_store[i+1+offset][j-1] * (Poly_n<Type>(vector<Type>{T(i+j), -1}) / (T(i+j)-T(i)));
		}
		_B_Spline_Base_build = 1;
	}
\end{lstlisting}

B 样条求解过程:

\begin{lstlisting}[language=C++]
    void build(vector<tuple<bool, int, Type>> bc= vector<tuple<bool, int, Type>>(0)) {
        if(this->_spline_builded) { return; }
        if(!this->_self_checked) { this->__self_check__(); this->_self_checked = 1; }
		if(bc.size() < Order - 1) {
			std::cerr << "B Spline : Too few extra conditions.\n";
			return;
		}
		sort(bc.begin(), bc.end());
		
		Mat<Type> A(N+Order, N+Order); Vec<Type> b(N+Order);

		for(int i = 0; i <= N; ++i) b[i] = Y[i];

        // 点值条件
		for(int j = 0; j <= N; ++j){
			for(int i = j-offset+1; i <= j; ++i) {
				A[j][i+offset-1] += get_B_Spline(i, Order)(j, T(j));
			}
		}

        // 额外边值条件
		for(int i = 0, j = N+1; i < Order-1; ++i, ++j) {
			if(i < Order-2 && get<0>(bc[i]) == get<0>(bc[i+1]) && 
				get<1>(bc[i]) == get<1>(bc[i+1])) {
					std::cerr << "B Spline : extra condition not satisfiable.\n";
					return;
				}
			else if (get<1>(bc[i]) <= 0 && get<1>(bc[i]) > Order){
				std::cerr << "B Spline : extra condition not satisfiable.\n";
				return;
			}
			else {
				b[j]= get<2>(bc[i]);
				int nd = get<1>(bc[i]);
				if(get<0>(bc[i])) { // right bounder
					for(int k = 0; k < Order; ++k)
						A[j][N-k+offset-1] += (get_B_Spline(N-k, Order).Derivative(nd))(N, T(N));
				} else { // left bounder
					for(int k = 0; k < Order; ++k)
						A[j][k] += (get_B_Spline(k-Order+1, Order).Derivative(nd))(0, T(0));
				}
			}
		}

        // 利用 linear.hpp 中的高斯消元函数
		a = Gauss_elimination(A, b).val;

        this->_spline_builded = 1;
    }
\end{lstlisting}

转化为 ppForm 形式:

\begin{lstlisting}[language=C++]
    B_Spline_Base ret = a[0] * get_B_Spline(-offset+1, Order);
	for(int i = -offset+2; i <= N; ++i) {
		ret = ret + a[i+offset-1] * get_B_Spline(i, Order);
	}
	auto all_splines = ret.f;
	for(int i = 0; i < N; ++i) splines[i] = all_splines[i + Order];
	_calculate_ppForm = 1;
\end{lstlisting}

\section{Cardinal B Spline 类}

\begin{lstlisting}[language=C++]
    template <class Type, int Order>
    class Cardinal_B_Spline : public Spline<Type>
\end{lstlisting}


包含成员如下:
\begin{enumerate}
    \item \LII{vector<Type> a}:B 样条系数。
    \item \LII{Type d, L, R}:样条左右边界,结点间隔;
    \item \LII{bool _calculate_ppForm}:是否已经将样条转化为分段多项式形式保存在 \LII{splines} 中;
    \item \LII{void buildLinearSpline()}:建立线性样条;
    \item \LII{void buildCubicSpline(...)}:建立三次样条;
    \item \LII{void build()}:样条建立接口;
    \item \LII{class Cardinal_B_Spline_Base}:B样条基函数类,支持加,乘多项式,数乘,平移变换,伸缩变换等运算,本质是维护相关分段多项式;
    \item \LII{vector<Cardinal_B_Spline_Base>}:B 样条基函数。
    \item \LII{const Cardinal_B_Spline_Base get_B_Spline(int i, int n) const}:返回真实下标的 $B_{t_I}^n$;
\end{enumerate}

构造函数:

\begin{lstlisting}[language=C++]
    Cardinal_B_Spline(const Type &_L, const Type& _R, const int &_n, 
		const vector<Type>& y) :L(_L),R(_R),d((_R-_L)/(_n-1)),
		Spline<Type>(evenspace<Type>(_L, _R, _n), y), _calculate_ppForm(0), a(_n-1+Order){}
	// evenspace 用于得到等间隔点列
\end{lstlisting}

求值函数:

\begin{lstlisting}[language=C++]
    Type operator() (const Type &x) const {
		custom_assert(this->_spline_builded, "Cardinal B Spline : Spline have not been built yet.");
        custom_assert(x >= X[0]-1e-10 && x <= X.back()+1e-10, "Cardinal B Spline : Input x is out of range.");
		Type ret = 0;
		int pos = (int)((x - L) / d);

		for(int j = max(-Order+1, pos-Order+1); j <= min(N, pos+1); ++j) {
			ret += a[j+Order-1]*get_B_Spline(j, Order)((x-L)/d);
		}
		return ret;
	}
\end{lstlisting}

Cardinal B 样条基函数计算,
因为具有平移不变性,所以不需要多每个下标都求解。

\begin{lstlisting}[language=C++]
    const Cardinal_B_Spline_Base get_B_Spline(int i, int n) const {
		static vector<Cardinal_B_Spline_Base> Cardinal_B_Spline_Base_store = {Cardinal_B_Spline_Base(0)};
		if (n < Cardinal_B_Spline_Base_store.size()) return Cardinal_B_Spline_Base_store[n].shift(i);
		for(int nn = Cardinal_B_Spline_Base_store.size(); nn <= n; ++nn) {
			Cardinal_B_Spline_Base_store.push_back(
				Cardinal_B_Spline_Base_store[nn-1] * (Poly_n<Type>(vector<Type>({1./nn,1./nn}))) + 
				Cardinal_B_Spline_Base_store[nn-1].shift(1)*(Poly_n<Type>(vector<Type>({1,-1./nn}))));
		}
		return Cardinal_B_Spline_Base_store[n].shift(i);
	}
\end{lstlisting}

样条建立过程:

\begin{lstlisting}[language=C++]
    /******************* begin Linear Spline *******************/
    void buildLinearSpline() { // build straightly 
        for(int i = 0; i <= N; ++i) a[i] = Y[i];
    }
    /******************* end Linear Spline *******************/

    /******************* begin Cubic Spline *******************/
    void buildCubicSpline(Cubic_Spline_Condition<Type, Nature> cond) {
		Vec<Type> x(N), y(N+1), z(N), b(N+1);
        for(int i = 1; i < N; ++i) x[i] = 1; x[0] = 0;
		for(int i = 0; i < N-1; ++i) z[i] = 1; z[N-1] = 0;
		y[0] = y[N] = 6; for(int i = 1; i < N; ++i) y[i] = 4;
		for(int i = 0; i <= N; ++i) b[i] = 6*Y[i];
		auto _a = Thomas(y, z, x, b).val;
		for(int i = 1; i <= N+1; ++i) a[i] = _a[i-1];
		a[0] = 2*a[1]-a[2]; a[N+2] = 2*a[N+1]-a[N];
    }
    void buildCubicSpline(Cubic_Spline_Condition<Type, Complete> cond) {
		Vec<Type> x(N), y(N+1), z(N), b(N+1);
        for(int i = 0; i < N; ++i) x[i] = 1;
		for(int i = 0; i < N; ++i) z[i] = 1;
		y[0] = y[N] = 2; for(int i = 1; i < N; ++i) y[i] = 4;
		for(int i = 1; i < N; ++i) b[i] = 6*Y[i];
		b[0] = 3*Y[0] + 1/(d)*cond.sa;
		b[N] = 3*Y[N] - 1/(d)*cond.sb;
		auto _a = Thomas(y, z, x, b).val;
		for(int i = 1; i <= N+1; ++i) a[i] = _a[i-1];
		a[0] = a[2] - 2/d*cond.sa;
		a[N+2] = a[N] + 2/d*cond.sb;
    }
    void buildCubicSpline(Cubic_Spline_Condition<Type, Second_Derivatives> cond) {
		Vec<Type> x(N), y(N+1), z(N), b(N+1);
        for(int i = 1; i < N; ++i) x[i] = 1; x[0] = 0;
		for(int i = 0; i < N-1; ++i) z[i] = 1; z[N-1] = 0;
		y[0] = y[N] = 6; for(int i = 1; i < N; ++i) y[i] = 4;
		for(int i = 0; i <= N; ++i) b[i] = 6*Y[i];
		b[0] -= 1/(d*d)*cond.sa;
		b[N] -= 1/(d*d)*cond.sb;
		auto _a = Thomas(y, z, x, b).val;
		for(int i = 1; i <= N+1; ++i) a[i] = _a[i-1];
		a[0] = 2*a[1]-a[2]+1/(d*d)*cond.sa; a[N+2] = 2*a[N+1]-a[N]+1/(d*d)*cond.sb;
    }
    /******************* end Cubic Spline *******************/

    void build() {
        if(this->_spline_builded) { return; }
        if(!this->_self_checked) { this->__self_check__(); this->_self_checked = 1; }
        if(Order == 1) buildLinearSpline();
        else { std::cerr << "B Spline: build() not implemented.\n"; return; }
        this->_spline_builded = 1;
    }

    template <Cubic_Spline_Type cType>
    void build(Cubic_Spline_Condition<Type, cType> cond) {
        if(this->_spline_builded) { return; }
        if(!this->_self_checked) { this->__self_check__(); this->_self_checked = 1; }
        if(Order == 3) {
			// construct linear system Ax = b
			buildCubicSpline(cond);
        }
        else { std::cerr << "Cardinal_B_Spline: build() not implemented.\n"; return; }
        this->_spline_builded = 1;
    }


\end{lstlisting}

转化为 ppForm 格式:

\begin{lstlisting}[language=C++]
    Cardinal_B_Spline_Base ret = a[0] * get_B_Spline(-Order+1, Order);
    for(int i = -Order+2; i <= N; ++i) {
        ret = ret + a[i+Order-1] * get_B_Spline(i, Order);
    }
    auto all_splines = ret.changescale(d).shift(L);
    for(int i = 0; i < N; ++i) splines[i] = all_splines.f[i + Order];
    _calculate_ppForm = 1;
\end{lstlisting}

\section{Quadratic Cardinal Spline 类}

由于 Quadratic Cardinal Spline 的插值逻辑与别的样条很不一样,因此考虑单独设计。

\begin{lstlisting}[language=C++]
    template <class Type>
    class Quadratic_Cardinal_B_Spline : public Spline<Type>
\end{lstlisting}

包含成员如下:
\begin{enumerate}
\item \LII{int L, R, n}:样条左右端点,左右端点距离;
\item \LII{vector<Type> a}:B 样条系数;
\item \LII{Type B(int i, Type x)}:得到 $B_i^2(x)$;
\item \LII{Poly_n<Type> getB(int seg/*0, 1, 2*/, int i)}:获得 $B_i^2$ 第 seg 段多项式。
\end{enumerate}

构造函数:

\begin{lstlisting}[language=C++]
    Quadratic_Cardinal_B_Spline(const int& x0, const int& _n, const vector<Type>& f, const Type& f0, const Type& fn) 
	: L(x0), R(x0 + _n), n(_n), _calculate_ppForm(0) {
		this->X.resize(n+1);
		for(int i = 0; i <= n; ++i) this->X[i] = L + i;
		Vec<Type> x(n), y(n-1), z(n-1), b(n);
		for (int i = 1; i < n-1; ++i) {
			x[i] = 6; y[i-1] = 1;
			z[i] = 1; b[i] = 8 * f[i];
		}
		x[0] = 5, z[0] = 1, b[0] = 8 * f[0] - 2 * f0;
		x[n-1] = 5, y[n-2] = 1, b[n-1] = 8 * f[n-1] - 2 * fn;
		vector<Type> t = Thomas(x, y, z, b).val;
		a.resize(n+2);
		for (int i = 0; i < n; ++ i) a[i+1] = t[i];
		a[0] = 2 * f0 - a[1];
		a[n+1] = 2 * fn - a[n];
		this->_spline_builded = 1;
	}
\end{lstlisting}

求值函数:

\begin{lstlisting}[language=C++]
    Type operator() (const Type& x) const {
		custom_assert(x >= L-1e-10 && x <= R+1e-10, "Quadratic_Cardinal_B_Spline : Input x out of range.");
		int i = floor(x);
		Type res = 0;
		if(i-1 >= L-1 && i-1 <= R) res += a[i-1-L+1] * B(i-1, x);
		if(i   >= L-1 && i   <= R) res += a[i  -L+1] * B(i  , x);
		if(i+1 >= L-1 && i+1 <= R) res += a[i+1-L+1] * B(i+1, x);
		return res;
	}
\end{lstlisting}

转化为 ppForm 格式:
\begin{lstlisting}[language=C++]
    splines.resize(n);
	for(int i = 0; i < n; ++i) for(int j = i; j <= i + 2; ++j) {
 		splines[i] += getB(2-j+i, j-1+L) * a[j];
	}
	_calculate_ppForm = 1;
\end{lstlisting}

外部接口:

\begin{lstlisting}[language=C++]
    template <class Type>
    Quadratic_Cardinal_B_Spline<Type> 
    Quadratic_Cardinal_B_Spline_Interpolation(const Function<Type>& f, const int& l, const int& n) {
    	vector<Type> y(n);
    	for(int i = 0; i < n; ++ i) y[i] = f(l+i+0.5);
    	Type f0 = f(l), fn = f(l+n);
    	return Quadratic_Cardinal_B_Spline<Type>(l, n, y, f0, fn);
    }
\end{lstlisting}

\section{Curve fitting 类}

\begin{lstlisting}[language=C++]
    class Curve_fitting_Order3 { // Here I use B_spline
	B_Spline<NUM, 3> x, y; // x 与 y 分别建立样条。
    public:
    	Curve_fitting_Order3 (vector<NUM> &_t, vector<NUM> &_x, vector<NUM> &_y) : x(_t, _x), y(_t, _y) {}
    	void buildNature() {
    		x.build({{0, 2, 0}, {1, 2, 0}});
    		y.build({{0, 2, 0}, {1, 2, 0}});
    	}
    	void buildComplete(NUM xa, NUM xb, NUM ya, NUM yb) {
    		x.build({{0, 1, xa}, {1, 1, xb}});
    		y.build({{0, 1, ya}, {1, 1, yb}});	
    	}
    	void buildSecondDerivatives(NUM xa, NUM xb, NUM ya, NUM yb) {
    		x.build({{0, 2, xa}, {1, 2, xb}});
    		y.build({{0, 2, ya}, {1, 2, yb}});	
    	}
    	const std::string to_python () {
            return "X = " + x.to_python() + "\nY = " + y.to_python();
        }
    };
\end{lstlisting}



\nocite{*}
\printbibliography[heading=bibintoc, title=\ebibname]

% \appendix
% % \appendixpage
% \addappheadtotoc

\end{document}