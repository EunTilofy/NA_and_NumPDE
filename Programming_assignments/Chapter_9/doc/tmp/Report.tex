\documentclass[lang=cn,a4paper,newtx,bibend=bibtex]{elegantpaper}
\usepackage{env}
\addbibresource[location=local]{reference.bib}

\title{\emph{《多重网格法解 Poisson 方程》}项目文档}
\author{张志心~~3210106357~~计科2106}
\date{Apr. 9. 2024}

\begin{document}
\maketitle

\tableofcontents
\newpage

\begin{abstract}

  该程序包为一维/二维正方形区域 Poisson 方程的求解器,
  采用多重网格法,支持 Dirichlet,Neumann,以及混合边值条件问题。
  
  该程序的时间复杂度为 $O(n)$,其中 $n$ 为网格的格点数。
  
  该程序包进行了充分的测试,对于一维/二维 Poisson 方程形式的边值
  问题,达到了二阶的收敛阶。
  
\end{abstract}

\section{用户文档}

该部分介绍用户使用此程序包求解一维/二维 Poisson 方程的方法。

\subsection{问题输入格式与规约}
用户需使用 \href{https://www.json.org/json-en.html}{JSON} 格式描述待求解问题(二维 Poisson 方程),
问题的数学形式为:
\[
    - \Delta u = f, \text{in~~} \Omega
\]

对于一维问题,$\Omega = [x_l, x_r]$,对于二维问题,
$\Omega = [x_l, x_r] \times [y_l, y_r] (x_r - x_l = y_r - y_l)$。

对于边值条件,有如下几种类型:
\begin{enumerate}
  \item Dirichlet 边值条件:$u = g, \text{on~~} \partial \Omega$;
  \item Nuemann 边值条件:$\bm{n} \cdot \nabla u = g, \text{on~~} \partial \Omega$;
  \item mixed 边值条件:$\begin{cases} u = g_1 & \text{on~~} X_1 \\ \bm{n} \cdot \nabla u = g_2 & \text{on~~} X_2\end{cases}, \bigg(X_1 \cap X_2 = \emptyset, X_1 \cup X_2 = \partial \Omega\bigg)$。
\end{enumerate}

\subsubsection{手动构造输入数据}

JSON 具体格式如下:
\begin{itemize}
  \item \lstinline{Dimension}:问题定义域的维数。
  \item \lstinline{Domain_Border}:问题定义域外边界,格式为 $[x_l, x_r]$(一维)或 $[x_l, x_r, y_l, y_r]$(二维),
  若用户未定义,则默认为 $[0, 1]$ 或 $[0, 1]^2$。
  对于二维情况,程序会检查输入外边界是否是一个合法的正方形。
  \item \lstinline{Domain_Type}:问题定义域的类型,包括 \lstinline{Regular}, \lstinline{Irregular}。
  \item \lstinline{Center}:若问题定义域是非规则类型,则需要提供内部挖去圆形的信息,圆心坐标格式为 $[x_c, y_c]$。
  \item \lstinline{R}:同上,圆形的半径为 $R$,程序会检查圆形是否位于外边界正方形的\textbf{内部}。
  \item \lstinline{Grid_n}:用户自定义网格的大小,程序会检查 $2 \le Grid\_n \le 248$,程序离散化的格点为 $(x_i, y_j) = (x_l + ih, y_l + jh), h = \frac1{Grid\_n}, i, j = 0, 1, \dots, Grid\_n$。
  \item \lstinline{BC_Type}:Poisson 方程的边值条件的类型,包括 \lstinline{Dirichlet}, \lstinline{Neumann},\lstinline{mixed}。
  \item \lstinline{Cycle_Type}:迭代方法,包括 V-Cycle(\texttt{V}) 和 FMG(\texttt{FMG})。
  \item \lstinline{Accuracy}:迭代终止的残差条件。
  \item \lstinline{Interpolation_opt}:插值算子采用的方法,包括 \lstinline{linear} 和 \lstinline{quadratic} 两种。
  \item \lstinline{Restriction_opt}:限制算子采用的方法,包括 \lstinline{injection} 和 \lstinline{full_weight} 两种。
  \item \lstinline{Max_Iteration}:最大迭代次数。
  \item \lstinline{initial_guess}:迭代的初始猜测,是一个函数,默认为 0。
  \item \lstinline{f}:描述 $- \Delta u = f, \text{in~~} \Omega$,
                       用 C++ 数学表达式的格式给出,表达式解析的具体方法见 \\
                       \texttt{/include/function\_generator/ExecCode2.hpp}。
  \item \lstinline{g}:描述 Dirichlet 边值条件($u = g, \text{on~~} \partial \Omega$)
                       或 Neumann 边值条件($\bm{n} \cdot \nabla u = g, \text{on~~} \partial \Omega$),
                       注意混合边值条件不使用该项来描述,
                       输入为一个 \texttt{JSON class},包含:
                       \begin{itemize}[$\circ$]
                        \item down : $y = y_l$ 上边值条件表达式;
                        \item left : $x = x_l$ 上边值条件表达式;
                        \item right : $x = x_r$ 上边值条件表达式;
                        \item up : $y = y_r$ 上边值条件表达式.
                       \end{itemize}
                       如果上述各部分的边值条件的表达式相同,用户可以使用 all 来描述。
  \item \lstinline{mixed_g}:仅用于描述混合边值条件,若非混合边值条件,则不需要该项。输入格式为一个 \texttt{JSON class}(与 \texttt{g} 类似),每个方向的键值为一个二元组,即 \{边值条件的类型, 边值条件的表达式\}。
  \item \lstinline{Need_Error}:表示是否需要进行解误差分析,为 boolean 类型(\texttt{true} 或 \texttt{false});
  \item \lstinline{answer}:若 \texttt{Need\_Error = true},需要提供问题的真解,输入格式同 f。
\end{itemize}
~~\\
下面给出部分输入数据作为示例:
\begin{enumerate}
\item 一维 Dirichlet 边值条件:
\begin{lstlisting}
  {
   "Accuracy" : 1.0000000000000001e-10,
   "BC_Type" : "Dirichlet",
   "Cycle_Type" : "FMG",
   "Dimension" : 1,
   "Domain_Border" : [ 0.0, 1.0 ],
   "Domain_Type" : "Regular",
   "Grid_n" : 16,
   "Interpolation_opt" : "linear",
   "Max_Iteration" : 50,
   "Need_Error" : true,
   "Restriction_opt" : "full_weight",
   "answer" : "exp(sin(x))",
   "initial_guess" : "0",
   "f" : "(sin(x)-cos(x)*cos(x))*exp(sin(x))",
   "g" : {
      "left" : "1",
      "right" : "exp(sin(1))"
   },
   "mixed_g" : null
  }
\end{lstlisting}

\item 二维 Neumann 边值条件:
\begin{lstlisting}
  {
    "Accuracy" : 1e-10,
    "BC_Type" : "Neumann",
    "Cycle_Type" : "FMG",
    "Dimension" : 2,
    "Domain_Border" : [ 0.0, 1.0, 0.0, 1.0 ],
    "Domain_Type" : "Regular",
    "Grid_n" : 32,
    "Interpolation_opt" : "quadratic",
    "Max_Iteration" : 50,
    "Need_Error" : true,
    "Restriction_opt" : "full_weight",
    "answer" : "exp(sin(x)+y)",
    "f" : "-(1-sin(x)+cos(x)*cos(x))*exp(sin(x)+y)",
    "g" : {
       "down" : "-exp(sin(x))",
       "left" : "-exp(y)",
       "right" : "cos(1)*exp(sin(1)+y)",
       "up" : "exp(sin(x)+1)"
    }
  }
\end{lstlisting}

\item 二维混合边值条件:
\begin{lstlisting}
  {
    "Accuracy" : 1e-10,
    "BC_Type" : "mixed",
    "Cycle_Type" : "FMG",
    "Dimension" : 2,
    "Domain_Border" : [ 0.0, 1.0, 0.0, 1.0 ],
    "Domain_Type" : "Regular",
    "Grid_n" : 32,
    "Interpolation_opt" : "linear",
    "Max_Iteration" : 50,
    "Need_Error" : true,
    "initial_guess" : "0",
    "Restriction_opt" : "injection",
    "answer" : "exp(sin(x)+y)",
    "f" : "-(1-sin(x)+cos(x)*cos(x))*exp(sin(x)+y)",
    "mixed_g" : {
       "down" : [ "Dirichlet", "exp(sin(x)+y)" ],
       "left" : [ "Dirichlet", "exp(sin(x)+y)" ],
       "right" : [ "Neumann", "cos(1)*exp(sin(1)+y)" ],
       "up" : [ "Neumann", "exp(sin(x)+1)" ]
    }
  }
\end{lstlisting}
\end{enumerate}

~~\\

\subsubsection{自动构造输入数据}

程序包提供了交互式输入数据,数据自检,并自动构造JSON文件的方法,
方法如下:
\begin{lstlisting}
  make dataGen
  ./dataGen
\end{lstlisting}

\subsubsection{数据自检}

程序会输入数据进行严格的检查,当输入数据不合法时,程序会报出异常并终止,返回 -1。
用户也可以自己测试数据是否满足条件,具体方法如下:
假设输入JSON数据的路径为 \lstinline{std::string file},

\begin{lstlisting}[language=C++]
  Multigrid_BVP_Problem new_Problem;
  deserialize_Json(new_Problem, file); 
  try {
      new_Problem._self_checked();
  } catch (char const * e) {
      cerr << e << "\n";
      // do anything you can
  }
\end{lstlisting}

该方法也用于用户自主创建问题实例,关于问题类的相关定义在 \ref{2.1} 中说明。

\subsection{问题求解}

\subsubsection{相关函数}

 程序调用 \lstinline{Multigrid_BVPsolver::solveProblem(std::string FILE, bool print)} 来求解问题,
 该函数是对求解流程的进一步封装。
 使用方法为:假设输入JSON数据的路径为 \lstinline{std::string file},
\begin{lstlisting}
  Square_BVPsolver solver;
  solver.solveProblem(file, /*print*/ 0);
\end{lstlisting}
  若 print = 1,则程序会在 \texttt{stderr} 中输出问题的描述和求解过程,
  若 print = 0,则程序只会输出最后的误差分析(若 Need\_error = 1)。

  用户也可以使用原始接口,具体如下:
\begin{lstlisting}[language=C++]
  Multigrid_BVPsolver::readProblem(const std::string file); // 读取数据并检查
  Multigrid_BVPsolver::printProblem() const; // 输出问题描述
  Multigrid_BVPsolver::solveProblem(bool print = 0); // 在readProblem之后调用,求解当前问题。
  Multigrid_BVPsolver::solveProblem(const Multigrid_BVP_Problem &_prob); // 求解已有的问题
  Multigrid_BVPsolver::Summary(); // 在solveProblem之后调用,输出当前问题和求解误差分析。
  Multigrid_BVPsolver::saveResults(std::string file = "res.csv"); // 输出问题求解结果。
\end{lstlisting}

\subsubsection{自动求解}

该程序包提供了 \texttt{test.cpp} 用于直接求解用户问题,
在根目录下编译求解程序:

\begin{lstlisting}
  make all
\end{lstlisting}

求解方法:
\begin{lstlisting}
  usage: ./test [-v|--verbose] <input JSON file>
\end{lstlisting}

其中 -v 或 --verbose 选项将会提供更多的求解信息(包括问题描述,求解过程,误差分析等),
否则,将只提供求解的误差分析。

用户可以一次性求解一个或多个输入数据:
\begin{lstlisting}
  ./test --verbose 1.json 2.json 3.json
\end{lstlisting}

\section{设计文档}

该部分说明此程序包的相关组件的逻辑结构。

\subsection{Multigrid\_BVP\_Problem 类及其序列化方法}\label{2.1}

\texttt{Multigrid\_BVP\_Problem} 类用于描述本程序包用于解决的问题实例、
初始化问题的各项参数、检查问题是否满足规约条件。

\begin{lstlisting}[language = C++]
  class Multigrid_BVP_Problem{
    public:
        std::string BC_Type; /* boundary condition : mixed, Dirichlet, Neumann*/
        std::string Domain_Type; /* regular(default) irregular*/
        vector<NUM> Domain_Border; /* [xl, xr, yl, yr] (default : [0,1]x[0,1])*/
        NUM xl, xr, yl, yr, h;
        void getDomain();
        int Grid_n; /* Grid_n x Grid_n */
        std::string f; /* - \Delta u (in Domain)*/
        func F;
    
        // calculate error
        bool Need_Error;
        std::string answer;
        func Answer;

        std::string initial_guess;
        func Initial_guess;
    
        // Boundary Condition
        // Dirichlet / Neumann
        map<std::string, std::string> g; /* left, right, up, down -> function*/
                                         /* 0,    1,     2,  3*/
        vector<func> G;
        // mixed
        map<std::string, pair<std::string, std::string>> mixed_g; /* left, right, up, down -> function*/
                                                                  /* 0,    1,     2,  3 */
        vector<std::string> _bc_types;
    
        int Dimension;
        std::string Cycle_Type; // FMG or V
        std::string Restriction_opt; // full_weight, linear
        std::string Interpolation_opt; // lenear, quadratic
        int Max_Iteration;
        NUM Accuracy;
    
        Multigrid_BVP_Problem() : 
            Domain_Border({0, 1, 0, 1}), Domain_Type("Regular"), 
            Max_Iteration(20), Accuracy(1e-10), Need_Error(0) {}
      
        void _self_checked ();
        void print() const;
    };
\end{lstlisting}

该类实例可以直接序列化为 JSON 文档,或由 JSON 文档里反序列化得到。
此类用户自定义类的序列化方法在 \texttt{/include/serialization/serialize\_json.hpp} 中实现。
采用编译预处理方式展开该方法:

\begin{lstlisting}[language = C++]
REGISTER_SERIALIZATION_JSON(BC_Type, Domain_Type, Domain_Border, Dimension, 
                             Restriction_opt, Interpolation_opt, Cycle_Type, Grid_n, 
                             f, g, mixed_g, Need_Error, answer, Max_Iteration, Accuracy);
\end{lstlisting}

序列化:

\begin{lstlisting}[language = C++]
  serialize_Json(const Multigrid_BVP_Problem& prob, std::string filename);
\end{lstlisting}

反序列化:

\begin{lstlisting}[language = C++]
  deserialize_Json(Multigrid_BVP_Problem& prob, std::string filename);
\end{lstlisting}

\subsection{Multigrid\_BVP\_Solver 类}

\texttt{Multigrid\_BVP\_Solver} 类用于求解 Poisson 方程的边值问题,
其中保存有一个 \texttt{Multigrid\_BVP\_Problem prob} 表示当前求解的
问题,它可以调用两个类,\texttt{class Regular\_Multigrid\_BVPsolver;\\
class Irregular\_Multigrid\_BVPsolver;} ,为两种方程的求解器(均为模板类,
模板参数为 \texttt{int dim},表示求解问题的维数)。
求解器类中均包含 \texttt{void solve();} 和 \text{void ErrorAnalysis();}
两个成员函数,用于求解问题实例和误差分析。最终求解结果将保存在 \lstinline{results} 中。

\begin{lstlisting}[language = C++]
  class Multigrid_BVPsolver {
    Multigrid_BVP_Problem prob;
    RES<1> results1;
    RES<2> results2;
public:
    void readProblem(const std::string file);
    void printProblem() const;
    void solveProblem(bool print = 0);
    void solveProblem(const Multigrid_BVP_Problem &_prob);
    void solveProblem(std::string File, bool print = 0);
    void saveResults(std::string file = "res.csv") const;
    NUM norm1, norm2, normi;
    void Summary();
};
\end{lstlisting}

\subsection{template<int dim> Regular\_Multigrid\_BVPsolver 类}

\subsubsection{限制算子}

对于 $n$ 和一个 $(n+1)^d$ 维的向量,计算其在粗网格上的值(为一个 $(n/2+1)^d$ 维向量)。

一维情况:
\begin{lstlisting}[language=C++]
  Vec<NUM> Regular_Multigrid_BVPsolver<1>::Restriction(const Vec<NUM>& x, const int N) {
        Vec<NUM> ret;

        ret.resize((N>>1)+1);
        if(prob.Restriction_opt == "injection") {
            for(int i = 0; i < ret.size; ++i) {
                ret[i] = x[i<<1];
            }
        } else { // prob.Restriction_opt == "full_weight"
            ret[0] = x[0]; ret[N>>1] = x[N];
            for(int i = 1; i < ret.size-1; ++i) {
                ret[i] = 0.25*(x[(i<<1)-1]+2*x[i<<1]+x[i<<1|1]);
            }
        }
        return ret;
    }
\end{lstlisting}

二维情况:
\begin{lstlisting}[language=C++]
  Vec<NUM> Regular_Multigrid_BVPsolver<2>::Restriction(const Vec<NUM>& x, const int N) {
        Vec<NUM> ret;

        ret.resize((N>>1)+1);
        if(prob.Restriction_opt == "injection") {
            for(int i = 0; i < ret.size; ++i) {
                ret[i] = x[i<<1];
            }
        } else { // prob.Restriction_opt == "full_weight"
            ret[0] = x[0]; ret[N>>1] = x[N];
            for(int i = 1; i < ret.size-1; ++i) {
                ret[i] = 0.25*(x[(i<<1)-1]+2*x[i<<1]+x[i<<1|1]);
            }
        }
        return ret;
    }
\end{lstlisting}


\subsubsection{插值算子}

对于 $n$ 和一个 $(n+1)^d$ 维的向量,计算其在细网格上的值(为一个 $(2n+1)^d$ 维向量)。

一维情况:

考虑二次插值的方法:
对于网格上的点,直接取原先的值;对于其他节点,若其位于线段两侧,采用最近的三个粗网格上点进行
插值,否则,采用最近的对称的四个点进行插值。

\begin{lstlisting}[language=C++]
  Vec<NUM> Regular_Multigrid_BVPsolver<1>::Interpolation(const Vec<NUM>& x, const int N) {
        Vec<NUM> ret;
        
        ret.resize(N<<1|1);
        if(prob.Interpolation_opt == "linear") {
            for(int i = 0; i < ret.size; ++i) {
                if(i&1) ret[i] = 0.5*(x[i>>1] + x[(i>>1)+1]);
                else ret[i] = x[i>>1];
            }
        } else { // prob.Interpolation_opt == "quadratic"
            for(int i = 0; i < ret.size; ++i) {
                if(i&1) {
                    int j = i/2;
                    if(i == 1) ret[i] = (3*x[j]+6*x[j+1]-x[j+2])/8.0;
                    else if(i == (N<<1)-1) ret[i] = (3*x[j+1]+6*x[j]-x[j-1])/8.0;
                    else ret[i] = (-x[j-1]+9*x[j]+9*x[j+1]-x[j+2])/16.0;
                }
                else ret[i] = x[i>>1];
            }
        }
        return ret;
  }
\end{lstlisting}

二维情况:

对于网格上的点,直接取原先的值;对于网格线上但不在网格上的点,采用与一维二次插值相同的方法;
对于不在网格线上的点,考虑它四周的四个节点,设其左上角为 $(I, J)$,考虑以下四组插值点:
\begin{itemize}
  \item (I, J+1), (I+1, J), (I+1, J+1), (I, J+2), (I+1, J+2);
  \item (I, J), (I+1, J+1), (I, J+1), (I-1, J), (I-1, J+1);
  \item (I, J+1), (I+1, J), (I, J), (I, J-1), (I+1, J-1);
  \item (I, J), (I+1, J+1), (I+1, J), (I+2, J), (I+2, J+1).
\end{itemize}
对于每一组点,如果其在网格内,则对其进行一个插值,最后该点上的值是所有可以取到的插值组得到的值的平均值。


\begin{lstlisting}[language=C++]
  Vec<NUM> Regular_Multigrid_BVPsolver<2>::Interpolation(const Vec<NUM>& x, const int N) {
        Vec<NUM> ret;
        int N2 = N<<1;
        ret.resize((N2+1)*(N2+1));
        auto ID = [](int i, int j, int N) -> int { 
            return j * (N+1) + i; 
        }; // for 2d function
        if(prob.Interpolation_opt == "linear") {
            for(int i = 0; i <= N2; ++i)
            for(int j = 0; j <= N2; ++j) {
                int I = i>>1, J = j>>1;
                if((i&1) && (j&1)) ret[ID(i, j, N2)] = 0.25 * (
                    x[ID(I, J, N)] + x[ID(I+1, J, N)] + x[ID(I+1, J+1, N)] + x[ID(I, J+1, N)]);
                else if(i&1) ret[ID(i, j, N2)] = 0.5 * (x[ID(I, J, N)] + x[ID(I+1, J, N)]);
                else if(j&1) ret[ID(i, j, N2)] = 0.5 * (x[ID(I, J, N)] + x[ID(I, J+1, N)]);
                else ret[ID(i, j, N2)] = x[ID(I, J, N)];
            }
        } else { // prob.Interpolation_opt == "quadratic"
            for(int i = 0; i <= N2; ++i)
            for(int j = 0; j <= N2; ++j) {
                if((~i&1)&&(~j&1)) ret[ID(i, j, N2)] = x[ID(i>>1, j>>1, N)];
                else if(~i&1) {
                    int I = i>>1, J = j>>1;
                    if(j == 1) ret[ID(i, j, N2)] = (3*x[ID(I, J, N)]+6*x[ID(I, J+1, N)]-x[ID(I, J+2, N)])/8.0;
                    else if(j == N2-1) ret[ID(i, j, N2)] = (3*x[ID(I, J+1, N)]+6*x[ID(I, J, N)]-x[ID(I, J-1, N)])/8.0;
                    else ret[ID(i, j, N2)] = (-x[ID(I, J-1, N)]+9*x[ID(I, J, N)]+9*x[ID(I, J+1, N)]-x[ID(I, J+2, N)])/16.0;
                } else if(~j&1) {
                    int I = i>>1, J = j>>1;
                    if(i == 1) ret[ID(i, j, N2)] = (3*x[ID(I, J, N)]+6*x[ID(I+1, J, N)]-x[ID(I+2, J, N)])/8.0;
                    else if(i == N2-1) ret[ID(i, j, N2)] = (3*x[ID(I+1, J, N)]+6*x[ID(I, J, N)]-x[ID(I-1, J, N)])/8.0;
                    else ret[ID(i, j, N2)] = (-x[ID(I-1, J, N)]+9*x[ID(I, J, N)]+9*x[ID(I+1, J, N)]-x[ID(I+2, J, N)])/16.0;
                } else {
                    int I = i>>1, J = j>>1, cnt = 0, id = ID(i, j, N2);
                    if(J + 2 <= N) { cnt += 8;
                        ret[id] += (4*x[ID(I,J+1,N)]+4*x[ID(I+1,J,N)]+2*x[ID(I+1,J+1,N)]-x[ID(I,J+2,N)]-x[ID(I+1,J+2,N)]);
                    }
                    if(I - 1 >= 0) { cnt += 8;
                        ret[id] += (4*x[ID(I,J,N)]+4*x[ID(I+1,J+1,N)]+2*x[ID(I,J+1,N)]-x[ID(I-1,J,N)]-x[ID(I-1,J+1,N)]);
                    }
                    if(J - 1 >= 0) { cnt += 8;
                        ret[id] += (4*x[ID(I,J+1,N)]+4*x[ID(I+1,J,N)]+2*x[ID(I,J,N)]-x[ID(I,J-1,N)]-x[ID(I+1,J-1,N)]);
                    }
                    if(I + 2 <= N) { cnt += 8;
                        ret[id] += (4*x[ID(I,J,N)]+4*x[ID(I+1,J+1,N)]+2*x[ID(I+1,J,N)]-x[ID(I+2,J,N)]-x[ID(I+2,J+1,N)]);
                    }
                    ret[id] /= (NUM)cnt;
                }
            }
        }
        return ret;
    }
\end{lstlisting}

\subsubsection{V-Cycle}

V-cycles 的形式如下:
\[
  \bm{v}^h \leftarrow \text{VC} (\bm{v}^h, \bm{f}^h, \nu_1, \nu_2),
\]

\begin{enumerate}[(V-1)]
  \item 对 $A^h\uB^h = \bm{f}$ 在 $\Omega^h$ 上松弛 $\nu_1$ 次;
  \item 如果 $\Omega^h$ 是粗网格,转4,否则:
  \equ{
    \bm{f}^{2h} &\leftarrow I_h^{2h} (\bm{f}^h - A^h \bm{v}^h), \\
    \bm{v}^{2h} &\leftarrow \bm{0}, \
    \bm{v}^{2h} &\leftarrow \text{VC}^{2h}(\bm{v}^{2h}, \bm{f}^{2h}, \nu_1, \nu_2).
  }
  \item 校正:$\bm{v}^{h} \leftarrow \bm{v}^{h} + I_{2h}^h \bm{v}^{2h}$;
  \item 对 $A^h \bm{u}^h = \bm{f}^{h}$ 松弛 $\nu_2$ 次。
\end{enumerate}

\begin{lstlisting}[language=C++]
  void VC(Vec<NUM> &vh, const Vec<NUM> &fh, const int nu1, const int nu2, const int N) {
      int id = calid[N];
      cal[id].GoIter(vh, fh, nu1);
      if(N > 2) {
          Vec<NUM> f2h = Restriction((fh - cal[id].A * vh), N);
          Vec<NUM> v2h(prob.Dimension==1?N/2+1:(N/2+1)*(N/2+1));
          VC(v2h, f2h, nu1, nu2, N/2);
          vh = vh + Interpolation(v2h, N/2);
      }
      cal[id].GoIter(vh, fh, nu2);
  }
\end{lstlisting}

\subsubsection{FMG Cycle}

Full Multigrid V-Cycle 形式入下:
\[\mathbf{v}^h \leftarrow \texttt{FMG}^h(\mathbf{f}^h, \nu_1, \nu_2),\]

\begin{enumerate}[(F-1)]
\item  若 $\Omega^h$ 是粗网格,设置 $\mathbf{v}^h \leftarrow \mathbf{0}$,转3,否则
\equ{
 \mathbf{f}^{2h} &\leftarrow I_h^{2h}\mathbf{f}^{2h}, \\
 \mathbf{v}^{2h} &\leftarrow \texttt{FMG}^{2h}(\mathbf{f}^{2h}, \nu_1, \nu_2);
}
\item 校正:$\mathbf{v}^h \leftarrow \mathbf{v}^h + I_{2h}^h\mathbf{v}^{2h}$;
\item 执行一个 V 循环使用初始值 $\mathbf{v}^h$:$\mathbf{v}^h \leftarrow \texttt{VC}^h(\mathbf{v}^h, \mathbf{f}^h, \nu_1, \nu_2)$。
\end{enumerate}
\begin{lstlisting}[language=C++]
  void FMG(Vec<NUM> &vh, const Vec<NUM> &fh, const int nu1, const int nu2, const int N) {
      int id = calid[N];
      if(N > 2) {
          Vec<NUM> f2h = Restriction(fh, N);
          Vec<NUM> v2h(prob.Dimension==1?N/2+1:(N/2+1)*(N/2+1));
          FMG(v2h, f2h, nu1, nu2, N/2);
          vh = Interpolation(v2h, N/2);
      }
      VC(vh, fh, nu1, nu2, N);
  }
\end{lstlisting}

\subsubsection{求解过程}

首先根据 \texttt{initial\_guess} 设置初始 $u$ 的值,
每次迭代,使用当前的残差进行求解,并把结果加入 $u$ 中, 
若当前加入的向量的范数小于设置的值,则退出。
最后的 $b$ 就是当前的残差。

\begin{lstlisting}[language=C++]
  // 设置初始 Guess 向量 u
  b = b - cal[0].A * u;

  for(int iter = 0; iter < prob.Max_Iteration; ++iter) {
      Vec<NUM> v(len);
      if(prob.Cycle_Type == "V") {
          VC(v, b, 5, 5, n);
      } else {
          FMG(v, b, 5, 5, n);
      }
      u = u + v;
      b = b - cal[0].A * v;
      if(Norm2Vec(v) < prob.Accuracy) break;
  }
\end{lstlisting}

\subsection{其他组件}

\subsubsection{SparseMat 稀疏矩阵类}

本项目中涉及到的矩阵 $A$ 都是稀疏矩阵(即对于 $O(n^2)$ 级别的矩阵,非零项为 $O(n)$ 个)。
为了使矩阵乘向量的复杂度为 $O(n)$,我们优化了矩阵的存储方式,使用 $class SparseMat$。
见 (\lstinline{/include/linear.hpp})。
具体的,我们使用 \lstinline{MatLine} 来表示矩阵的一行,我们使用的 Hash 的方式,
将一行中,列下标模 5 余数相同的列保存在一起(这样是为了方便矩阵的随机访问),
在矩阵与向量的乘法中,直接依次取出矩阵中的非零元素与向量中的相应位置作乘法,并把其
贡献加入结果向量中。

\begin{lstlisting}[language=C++]
  template<class Type>
  class SparseMat {
      class MatLine { // 表示矩阵的一行
          int len;
          Type zero;
      public:
          vector<pair<int,Type>> a[5];
          MatLine(int _len = 0) : len(_len), zero(0) {}
          Type& operator [] (const int& pos);
          const Type& operator [] (const int& pos) const;
          void setv(int pos, Type v);
          Type v(int pos);
          Type indot(const Vec<Type>& b) const;
          MatLine operator * (const Type& x) const;
          MatLine operator - () const;
          MatLine operator / (const Type& x) const;
          friend ostream& operator << (ostream& o, const MatLine &line);
          operator Vec<Type>() const;
      };
      vector<MatLine> a;
  public:
      int d, d2; // the dimension of column vector and row vector.
      typedef Vec<Type> VecT;
      typedef Mat<Type, Type> MatT;
      SparseMat() : d(0), d2(0), a(0) {}
      SparseMat(const int& _d) : d(_d), d2(_d), a(_d, _d) {}
      SparseMat(const int& _d, const int& _d2) : d(_d), d2(_d2), a(_d, _d2) {}
      
      MatLine& operator[] (const int &x) { return a[x]; }
      const MatLine& operator[] (const int &x) const { return a[x]; }
      auto begin() const -> decltype(a.begin()) {
          return a.begin();
      }
      auto begin() -> decltype(a.begin()) {
          return a.begin();
      }
      auto end() const -> decltype(a.end()) {
          return a.end();
      }
      auto end() -> decltype(a.end()) {
          return a.end();
      }
  
      VecT operator * (const VecT& y) const;
      SparseMat<Type> operator * (const Type &x) const ;
      SparseMat<Type> operator - () const ;
      friend SparseMat<Type> operator * (const Type &x, const SparseMat<Type> &y);
      SparseMat<Type> operator / (const Type& x) const;
      operator MatT() const;
      void setv(int i, int j, Type v) { a[i].setv(j, v); }
      void v(int i, int j) { return a[i].v(j); }
  };
\end{lstlisting}

\subsubsection{带权 Jacobi 迭代}

定义 $D, -L, -U$ 为 $A$ 的对角,下三角,上三角部分($A = D - L - U$)。定义:
\[
  \Cases{
    T_J = D^{-1} (L + U), \\
    \bm{c} = D^{-1} \bB.
  }
\]
带权 Jacobi 迭代为如下形式的不动点迭代:
\equ{
  \xB_* &= T_j \xB^{(k)} + \bm{c},\\
  \xB^{k+1} &= (1 - \omega) \xB^{(k)} + \omega \xB_* = [(1 - \omega)I + \omega T_J]\xB^{(k)} + \bm{c}.
}
这里都取 $\omega = \frac23$。
我们使用 \lstinline{Jacobi_Iteration::GoIter(vec<NUM> &x, const Vec<NUM>&b, int iter = 1)} 表示
从 \lstinline{x} 开始,对于方程 $Ax = b$,迭代 \lstinline{iter} 次。
因为问题中对于每种网格的 $A$ 矩阵总是固定的,但是右端项总是不相同,因此考虑将 $b$ 作为参数输入,
而把 $A$ 作为类的成员。

\begin{lstlisting}[language=C++]
class Jacobi_Iteration {
    SparseMat<NUM> T, nD;
    Vec<NUM> c;
    NUM w;
    int n;
public:
    SparseMat<NUM> A;
    Jacobi_Iteration(const SparseMat<NUM>&_A, const int& _n, const NUM& _w = 2.0/3.0) : 
                     A(_A), w(_w), n(_n), T(-_A), nD(_n+1) {
                        for(int i = 0; i <= n; ++i) {
                            T[i][i] = 0;
                            nD[i][i] = 1.0 / _A[i][i];
                            T[i] = T[i] * nD[i][i];
                        }
                     }
    void GoIter(Vec<NUM> &x, const Vec<NUM>& b, int iter = 1) {
        c = nD * b;
        Vec<NUM> y;
        for(int i = 0; i < iter; ++i) {
            y = T * x + c;
            x = w * y + (1-w) * x;
        }
    }
};
\end{lstlisting}

\subsubsection{function类及字符串解析方法}

求解问题中需要用到的2元函数为 \texttt{/include/function.hpp} 中的 \texttt{Function2D} 类的派生类,
因为涉及到从 JSON 文档中读取函数的表达式,项目调用了 \texttt{/include/function\_generator/ExecCode2.hpp} 中
的 \texttt{FromString} 类,该类用于对一个字符串建立表达式树,
并在之后求解的时候可以快速根据表达式树进行求值。
对于多元函数,可以调用 \texttt{FromString<double, double>::cal2(const map<string, double>\& pars)} 进行求值。

\begin{lstlisting}[language = C++]
  class func2D : Function2D<double, double> {
    FromString<double, double> getvalue;
  public:
    func2D (const std::string &_str) : getvalue(_str) {}
    func2D () : getvalue() {}
    void set(const std::string &_str) { getvalue = FromString<double, double>(_str); }
    double operator() (const double &x, const double &y) const {
        double xx = x, yy = y;
        return getvalue.cal2({{"x", xx}, {"y", yy}});
    }
  };
\end{lstlisting}

\section{样例测试及结果}

该部分说明此程序包在样例测试中的求解表现。

\subsection{测试说明}

本程序包对于四个 Poisson 方程(三个一维方程,一个二维方程)关于不同的边值条件提供了若干测试数据,
测试数据位于目录 \texttt{/data} 下。

在根目录下编译并求解所有样例:
\begin{lstlisting}
  make all # compile
  make run
\end{lstlisting}

该指令将会编译 \texttt{/test.cpp},并且调用 \texttt{test} 按照字典序依次
运行所有的样例(\texttt{/data} 下的所有 JSON 文件),并且保存样例的误差分析结果至 \texttt{/res/test\_analysis.txt}。
测试输出(离散网格点值的求解结果)将会以表格的形式保存至 \texttt{/res/<Input File Name>.csv}。

对测试输出进行绘图:
\begin{lstlisting}
  make plot
\end{lstlisting}

该指令将会对保存在 \texttt{/res/} 目录下的所有 csv 输出文件进行绘图,
并且将结果保存在 \texttt{/res/} 下,命名与原文件相同,类型为 png。

所有测试共花费 1 min 左右,
程序包提供了一份测试结果的备份,位于 \texttt{/res\_bac} 目录,
另有一份所有样例的误差报告 \texttt{terminal.out},
便于用户直接查看。

因报告篇幅有限,下只展示部分测试结果的收敛性分析。

\subsection{测试样例1}

测试样例为 \texttt{data/1a-*.json},求解结果见 \texttt{terminal.out}。
该部分指定 Accuracy 为 1e-8。

问题: 
\[
  u(x) = \exp(\sin(x)), \Omega = [0, 1]
\]

\subsubsection{Dirichlet 边值条件}

\[
  \Cases{
  -\Delta u = \exp(\sin(x)) (\sin(x) - \cos^2(x)), \text{in} ~\Omega \\
  u = \exp(\sin(x)), \text{on}~\partial \Omega
  }
\]


\subsubsection{Neumann 边值条件}

\[
  \Cases{
  -\Delta u = \exp(\sin(x)) (\sin(x) - \cos^2(x)), \text{in} ~\Omega \\
  \apart{u}{\bm{n}}{} = -1, x = 0 \\
  \apart{u}{\bm{n}}{} = \exp(\sin(1))\cos(1), x = 1
  }
\]

\subsubsection{混合边值条件}

\[
  \Cases{
  -\Delta u = \exp(\sin(x)) (\sin(x) - \cos^2(x)), \text{in} ~\Omega \\
  \apart{u}{\bm{n}}{} = -1, x = 0 \\
  u = \exp(\sin(1)), x = 1
  }
\]

\subsubsection{V-Cycle和一阶算子的测试}

一维Dirichlet边界条件,V-Cycle,linear插值,injection限制:

\begin{longtable}{|c|cccc|} \hline
网格大小 & 残差 & 误差 & 运行时间(ms) & 收敛阶 \\ \hline
32 	& 1.00e-10	& 4.94e-05	& 0.247  & - \\ \hline
64 	& 2.00e-10	& 1.24e-05	& 0.347  & 1.99 \\ \hline
128	& 1.00e-09	& 3.09e-06	& 0.551  & 2.00 \\ \hline
256	& 3.80e-09	& 7.73e-07	& 0.910  & 2.00 \\ \hline
512	& 1.54e-08	& 1.93e-07	& 1.472  & 2.00 \\ \hline
\end{longtable}

一维Neumann边界条件,V-Cycle,linear插值,injection限制:

\begin{longtable}{|c|cccc|} \hline
网格大小 & 残差 & 误差 & 运行时间(ms) & 收敛阶 \\ \hline
32 	& 2.22e-03	& 4.08e-04	& 0.542  & - \\ \hline
64 	& 6.91e-04	& 1.02e-04	& 0.752  & 2.00 \\ \hline
128	& 2.15e-04	& 2.54e-05	& 1.078  & 2.00 \\ \hline
256	& 6.66e-05	& 6.35e-06	& 1.725  & 2.00 \\ \hline
512	& 1.59e-05	& 1.59e-06	& 2.911  & 2.00 \\ \hline
\end{longtable}

一维 Mixed 边界条件,V-Cycle,linear插值,injection限制:

\begin{longtable}{|c|cccc|} \hline
网格大小 & 残差 & 误差 & 运行时间(ms) & 收敛阶 \\ \hline
32 	& 1.26e-08	& 1.63e-04	& 0.441  & - \\ \hline
64 	& 7.80e-09	& 4.34e-05	& 0.634  & 1.91 \\ \hline
128	& 1.80e-08	& 1.12e-05	& 0.885  & 1.95 \\ \hline
256	& 3.88e-08	& 2.84e-06	& 1.413  & 1.98 \\ \hline
512	& 2.04e-08	& 7.16e-07	& 2.536  & 1.99 \\ \hline
\end{longtable}

\subsubsection{FMG-Cycle和二阶算子的测试}

一维Neumann边界条件,FMG,quadratic插值,Full-weight限制:

\begin{longtable}{|c|cccc|} \hline
网格大小 & 残差 & 误差 & 运行时间(ms) & 收敛阶 \\ \hline
32 	& 2.25e-03	& 4.07e-04	& 1.115  & - \\ \hline
64 	& 7.06e-04	& 1.02e-04	& 1.799  & 2.00 \\ \hline
128	& 2.20e-04	& 2.54e-05	& 2.606  & 2.00 \\ \hline
256	& 6.86e-05	& 6.34e-06	& 4.124  & 2.00 \\ \hline
512	& 2.13e-05	& 1.58e-07	& 6.643  & 2.00 \\ \hline
\end{longtable}

\subsubsection{总结}

多重网格算法具有二阶的收敛阶。
且在Dirichlet和Mixed边界条件下均可使残差降到$10^{-10}$;
在Neumann边界条件下残差收敛到定值,
即最小二乘问题$\min\Vert Au-f\Vert$的解的残差。

在一维问题中,网格较粗的情况下,
FMG-Cycle和高阶算子的收敛速度优势并不显著。
又因为V-Cycle和低阶算子的计算量明显小于高阶算子,
所以第一小节的测试结果明显优于第二小节。

在最细的网格上,将 Accuracy 逐渐降至 $1 \time 10^{-16}$,设置最大迭代次数为 100 (足够大)。
执行命令:
\begin{lstlisting}
  ./test data/1a-DFqf-512.json data/1a-DFqf-512-2.json data/1a-DFqf-512-3.json 
         data/1a-DFqf-512-4.json data/1a-DFqf-512-5.json
\end{lstlisting}
得到结果如下:
\begin{longtable}{|c|cccc|} \hline
  网格大小 & Accuracy & 残差 & 误差 & 运行时间(ms)   \\ \hline
  512 	 & 1e-8 & 3e-10	& 1.932e-7	& 4.255   \\ \hline
  512 	 & 1e-10 & 3e-10	& 1.932e-7	& 3.791   \\ \hline
  512	& 1e-12 & <1e-10	& 1.932e-7	& 3.771   \\ \hline
  512	& 1e-14 & <1e-10	& 1.932e-7	& 3.279   \\ \hline
  512	& 1e-16 & <1e-10	& 1.932e-7	& 3.643   \\ \hline
  \end{longtable}
可以发现,虽然残差仍然在下降,但是误差保持不变,这是因为,当 $n = 512$ 时,
由于 $h^2 = \frac{1}{262144} \approx 4e-6$,此时误差主要来自于截断误差,所以减小对残差的限制无法进一步减少总体误差。

\subsection{测试样例2}

测试样例为 \texttt{data/1b-*.json},求解结果见 \texttt{terminal.out}。
该部分指定 Accuracy 为 1e-12。

问题: 
\[
  u(x) = \ln(1 + x^2), \Omega = [0, 1]
\]

\subsubsection{Dirichlet 边值条件}

\[
  \Cases{
  -\Delta u =  -2(1-x^2)/(1+x^2)^2, \text{in} ~\Omega \\
  u = \log(1 + x^2), \text{on}~\partial \Omega
  }
\]

\subsubsection{Neumann 边值条件}

\[
  \Cases{
  -\Delta u =  -2(1-x^2)/(1+x^2)^2, \text{in} ~\Omega \\
  \apart{u}{\bm{n}}{} = 0, x = 0 \\
  \apart{u}{\bm{n}}{} = 1, x = 1
  }
\]

\subsubsection{混合边值条件}

\[
  \Cases{
  -\Delta u =  -2(1-x^2)/(1+x^2)^2, \text{in} ~\Omega \\
  \apart{u}{\bm{n}}{} = 0, x = 0 \\
  u = \ln(2), x = 1
  }
\]

\subsubsection{V-Cycle 和一阶算子的测试}

一维Mixed边界条件,V-Cycle,linear插值,injection限制:

\begin{longtable}{|c|cccc|} \hline
网格大小 & 残差 & 误差 & 运行时间(ms) & 收敛阶 \\ \hline
32 	& <1e-10	& 7.57e-05	& 1.115  & - \\ \hline
64 	& <1e-10	& 2.95e-05	& 1.799  & 1.36 \\ \hline
128	& <1e-10	& 8.76e-06	& 2.606  & 1.75 \\ \hline
256	& <1e-10	& 2.37e-06	& 4.124  & 1.89 \\ \hline
512	& <1e-10	& 6.14e-07	& 6.643  & 1.95 \\ \hline
\end{longtable}

\subsection{测试样例3}

测试样例为 \texttt{data/1c-*.json},求解结果见 \texttt{terminal.out}。
该部分指定 Accuracy 为 1e-12。

问题: 
\[
  u(x) = \sin^2(x), \Omega = [0, 1]
\]

\subsubsection{Dirichlet 边值条件}

\[
  \Cases{
  -\Delta u = -2 \cos(2x), \text{in} ~\Omega \\
  u = \sin^2(x), \text{on}~\partial \Omega
  }
\]




\subsubsection{Neumann 边值条件}

\[
  \Cases{
  -\Delta u = -2 \cos(2x), \text{in} ~\Omega \\
  \apart{u}{\bm{n}}{} = 0, x = 0 \\
  \apart{u}{\bm{n}}{} = \sin(2), x = 1
  }
\]



\subsubsection{混合边值条件}

\[
  \Cases{
  -\Delta u = \exp(\sin(x)) (\sin(x) - \cos^2(x)), \text{in} ~\Omega \\
  \apart{u}{\bm{n}}{} = 0, x = 0 \\
  u = \sin^2(1), x = 1
  }
\]




\subsection{测试样例4}

测试样例为 \texttt{data/2-*.json},求解结果见 \texttt{terminal.out}。
该部分指定 Accuracy 为 1e-12。

\[
  u(x, y) = \exp (y + \sin x), \Omega = [0, 1]^2
\]

\subsubsection{Dirichlet 边值条件}

\[
  \begin{cases}
    - \Delta u = -(1 - \sin x + \cos^2 x) \exp(\sin x + y), \text{in~~} \Omega \\
    u = \exp(\sin x + y), \text{on~~} \partial \Omega 
  \end{cases}
\]




\subsubsection{Neumann 边值条件}

\[
  \begin{cases}
    - \Delta u = -(1 - \sin x + \cos^2 x) \exp(\sin x + y), \text{in~~} \Omega \\
    \apart{u}{\bm{n}}{} = -\exp(\sin x), y = 0 \\
    \apart{u}{\bm{n}}{} = -\exp(y), x = 0 \\
    \apart{u}{\bm{n}}{} = \cos 1 \cdot \exp(\sin 1 + y), x = 1 \\ 
    \apart{u}{\bm{n}}{} = \exp(\sin x + 1), y = 1
  \end{cases}
\]



\subsubsection{混合边值条件}

\[
  \begin{cases}
    - \Delta u = -(1 - \sin x + \cos^2 x) \exp(\sin x + y), \text{in~~} \Omega \\
    u = \exp(\sin x + y), y = 0 \\
    u= \exp(\sin x + y), x = 0 \\
    \apart{u}{\bm{n}}{} = \cos 1 \cdot \exp(\sin 1 + y), x = 1 \\
    \apart{u}{\bm{n}}{} = \exp(\sin x + 1), y = 1
  \end{cases}
\]

\subsubsection{V-Cycle和一阶算子的测试}

二维Dirichlet边界条件,V-Cycle,linear插值,injection限制:

\begin{longtable}{|c|ccccc|} \hline
网格大小 & 残差 & 误差 & 运行时间(ms) & 收敛阶 & 运行时间收敛阶 \\ \hline
32 	& <1e-10	& 3.45e-05	& 3.753  & - & - \\ \hline
64 	& <1e-10	& 8.62e-06	& 15.811  & 2.00 & 2.07 \\ \hline
128	& <1e-10	& 2.16e-06	& 71.655 & 2.00 & 2.18 \\ \hline
256	& <1e-10	& 5.39e-07	& 303.294 & 2.00 & 2.08 \\ \hline
512	& <1e-10	& 1.35e-07	& 1289.425 & 2.00 & 2.09 \\ \hline
\end{longtable}

二维Neumann边界条件,V-Cycle,linear插值,injection限制:

\begin{longtable}{|c|ccccc|} \hline
网格大小 & 残差 & 误差 & 运行时间(ms) & 收敛阶 & 运行时间收敛阶 \\ \hline
32 	& 2.89e-03	& 8.56e-04	& 5.557  & - & - \\ \hline
64 	& 9.71e-04	& 2.16e-04	& 18.187  & 1.99 & 1.71 \\ \hline
128	& 3.49e-04	& 5.42e-05	& 105.937 & 1.99 & 2.54 \\ \hline
256	& 1.95e-04	& 1.37e-05	& 471.695 & 1.98 & 2.15 \\ \hline
512	& 2.86e-03	& 3.75e-06	& 1818.714 & 1.87 & 1.94 \\ \hline
\end{longtable}

二维Mixed边界条件,V-Cycle,linear插值,injection限制:

\begin{longtable}{|c|ccccc|} \hline
网格大小 & 残差 & 误差 & 运行时间(ms) & 收敛阶 & 运行时间收敛阶 \\ \hline
32 	& 7.81e-07	& 1.06e-03	& 4.795  & - & - \\ \hline
64 	& 4.38e-06	& 2.63e-03	& 17.392  & 2.01 & 1.86 \\ \hline
128	& 1.95e-05	& 6.58e-05	& 69.162 & 2.00 & 1.99 \\ \hline
256	& 7.36e-05	& 1.70e-05	& 336.728 & 1.95 & 2.28 \\ \hline
512	& 2.47e-04	& 5.52e-06	& 1598.827 & 1.62 & 2.24 \\ \hline
\end{longtable}

\subsubsection{FMG和二阶算子的测试}

测试函数$u(x,y)=\exp(\sin(x)+y)$,二维Mixed边界条件,FMG,linear插值,injection限制。

\begin{longtable}{|c|ccccc|} \hline
网格大小 & 残差 & 误差 & 运行时间(ms) & 收敛阶 & 运行时间收敛阶 \\ \hline
32 	& <1e-10	& 1.06e-03	& 5.499  & - & - \\ \hline
64 	& <1e-10	& 2.63e-03	& 17.582  & 2.01 & 1.67 \\ \hline
128	& <1e-10	& 6.54e-05	& 84.448 & 2.01 & 2.26  \\ \hline
256	& <1e-10	& 1.63e-05	& 342.296 & 2.00 & 2.03  \\ \hline
512	& <1e-10	& 4.07e-06	& 1420.509 & 2.00 & 2.05  \\ \hline
\end{longtable}

测试函数$u(x,y)=\exp(\sin(x)+y)$,二维Mixed边界条件,FMG,quadratic插值,Full-weight限制。

\begin{longtable}{|c|ccccc|} \hline
网格大小 & 残差 & 误差 & 运行时间(ms) & 收敛阶 & 运行时间收敛阶 \\ \hline
32 	& <1e-10	& 1.06e-03	& 5.341  & - & - \\ \hline
64 	& <1e-10	& 2.63e-03	& 18.999  & 2.01 & 1.83  \\ \hline
128	& <1e-10	& 6.54e-05	& 65.258 & 2.01 & 1.78  \\ \hline
256	& <1e-10	& 1.63e-05	& 314.449 & 2.00 & 2.27  \\ \hline
512	& <1e-10	& 4.07e-06	& 1865.895 & 2.00 & 2.57  \\ \hline
\end{longtable}

\subsubsection{分析与总结}

多重网格在二维问题中仍具有二阶的收敛阶,
且时间复杂度基本上和网格格点数量线性相关。

二维问题中,FMG-Cycle相比V-Cycle的优势可以体现:
二维的Neumann和mixed边界条件下,$n=128,256,512$时,V-Cycle在循环20次后残差没有收敛,
导致计算出的解收敛阶偏低,明显小于二阶。
而换成FMG-Cycle后,同样迭代20次,残差收敛至$10^{-10}$以下,解的收敛阶达到了二阶。

\nocite{*}
\printbibliography[heading=bibintoc, title=\ebibname]

\end{document}