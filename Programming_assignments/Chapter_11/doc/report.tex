\documentclass[lang=cn,a4paper,newtx,bibend=bibtex]{elegantpaper}
\usepackage{env}
\usepackage{tikz}
\usetikzlibrary{positioning}

\title{\emph{《初边值问题求解器》}项目文档}
\author{张志心~~3210106357~~计科2106}
\date{Jun. 17. 2024}

\begin{document}
\maketitle

\tableofcontents
\newpage

\section{用户文档}

\begin{verbatim}
    make all       # 编译该项目代码。
    make run       # 编译并运行该项目代码,并保存结果于 \texttt{/res}。
    make plot      # 绘制所有样例的图示 (对所有 res/*.csv 绘图)。
    make story     # 生成报告文档。
\end{verbatim}

\newpage
\section{数值方法说明}

线方法(Method of Lines, MOL)是一种求解初边值问题的方法。

它主要分两步完成。
第一步,在空间上离散,将初边值问题转化为关于 $U_j \approx u(\cdot,x_j)$ 的初值问题;
第二步,在时间上离散,运用初值问题的方法(Euler方法、Runge-Kutta方法等)求解。

在空间和时间上均建立均匀网格,步长分别为 $h=\frac 1m$ 和 $k=\frac TN$。

我们用 $(m+1)(N+1)$ 个离散点处的值 $U_j^n(0\leq j\leq m, 0\leq n\leq N)$ 来拟合解函数。

\subsection{Heat Equations}

带Dirichlet边界条件的一维空间上的热方程(one-dimensional heat equation)形如
\begin{eqnarray}
    u_t - \nu u_{xx} = & 0, & \forall (x,t)\in \Omega = (0,1)\times (0,T), \\
    u(x,0) = & f(x), & \forall x\in (0,1), \\
    u(0,t) = & g_0(t), & \forall t\in (0,T), \\
    u(1,t) = & g_1(t), & \forall t\in (0,T).
\end{eqnarray}

\subsubsection{空间离散}
热方程的空间离散格式为:

\begin{equation}
    U_j'(t) = \frac{\nu}{h^2}(U_{j-1}(t) - 2U_j(t) + U_{j+1}(t)), U_0(t) = g_0(t), U_1(t) = g_1(t).
\end{equation}

这是一个 $m-1$ 维空间上的常微分方程组 $\UB'(t) = A\UB(t) + \gB(t)$。

其中
\begin{equation}
    A=\frac {\nu}{h^2}
    \MAT{
        -2 & +1 & & & & \\
        +1 & -2 & +1 & & & \\
        & +1 & -2 & +1 & & \\
        & & \ddots & \ddots & \ddots & \\
        & & & +1 & -2 & +1 \\
        & & & & +1 & -2 \\
    },
    \UB(t) = \MAT{
        U_1(t) \\
        U_2(t) \\
        U_3(t) \\
        \vdots \\
        U_{m-2}(t) \\
        U_{m-1}(t) \\
    },
    \gB(t) = \MAT{
        g_0(t) \\
        0 \\
        0 \\
        \vdots \\
        0 \\
        g_1(t) \\
    }
\end{equation}

\subsubsection{时间离散1:$\theta-$方法}

$\theta-$ 方法就是将Euler方法应用于热方程的空间离散格式。
它是向前Euler方法和向后Euler方法的线性组合。具体格式为:
\begin{equation}
    \dfrac{U_j^{n+1}-U_j^n}k = \dfrac{\nu}{h^2}[\theta(U_{j-1}^{n+1}-2U_j^{n+1}+U_{j+1}^{n+1})+(1-\theta)(U_{j-1}^n-2U_j^n+U_{j+1}^n)].
\end{equation}
将上述方法整理,把上标为 $n+1$ 的项全部移到左侧,其他项移到右侧,即得到一个 $m+1$ 维稀疏线性方程组。

当 $\theta=0$ 时,$\theta-$ 方法就是FTCS方法;
$\theta=\frac 12$ 时,$\theta-$ 方法就是Crank-Nicoolson方法;
$\theta=1$ 时,$\theta-$ 方法就是BTCS方法。

$\theta-$ 方法是空间二阶、时间一阶收敛的。
特别地,当 $\theta$ 取 $\frac 12-\frac {h^2}{12k\nu}$ 时,可以达到空间四阶、时间二阶收敛。
当 $\theta\geq \frac 12$ 时,方法无条件稳定;
$\theta\in [0,\frac 12)$ 且 $k\leq \frac{h^2}{2(1-2\theta)\nu}$ 时,方法稳定。

\subsubsection{时间离散2:Gauss-Legendre RK 方法和组合 RK 方法}

将 RK 方法应用于热方程的离散格式为

\begin{eqnarray}
    y_j^{n,l} = & \frac 1{h^2}(U_{j-1}^n - 2U_j^n + U_{j+1}^n + k\sum_{r=1}^sa_{l,r}(y_{j-1}^{n,r} - 2y_j^{n,r} + y_{j+1}^{n,r})), \\
    U_j^{n+1} = & U_j^n + k\sum_{l=1}^s b_jy^{n,l}.
\end{eqnarray}

和 RK 方法应用于一般初值问题不同,我们在这里只需求解一个 $s(m+1)$ 维的稀疏线性方程组即可,而无需再使用不动点迭代。

Gauss-Legendre 方法是空间二阶、时间二阶收敛的。
Collocation RK 方法是空间二阶、时间三阶收敛的,且无条件稳定。

\subsection{Advection Equations}

带周期边界的一维空间上的平流方程(one-dimensional advection equations)形如
\begin{eqnarray}
    u_t + au_x = & 0, & \forall (x,t)\in \Omega = (0,1)\times (0,T), \\
    u(x,0) = & f(x), & \forall x\in (0,1), \\
    u(x+1, t) = & u(x, t). & \forall (x,t)\in \RBB\times (0,T). \\
\end{eqnarray}

平流方程的真解为
\begin{equation}
    u(x,t) = f(x-at).
\end{equation}

\subsubsection{LeapFrog 方法}

对 $u_t$ 和 $u_x$ 都使用双边离散,离散格式为
\begin{equation}
    \frac{U_j^{n+1} - U_j^{n-1}}{2k} = -a\frac{U_{j+1}^n - U_{j-1}^n}{2h}.
\end{equation}
注意这里下标的加减都是循环移位。下同。

这是一个直接法,无需求解方程组。注意当 $n=0$ 时因为 $n-1=-1$ 没有值,所以第一步使用单边导数离散代替:
\begin{equation}
    \frac{U_j^1 - U_j^0}{k} = -a\frac{U_{j+1}^0 - U_{j-1}^0}{2h}.
\end{equation}

\subsubsection{Lax-Friedrichs 方法}

对 $u_t$ 的离散改用单边离散,并用两边平均值代替 $U_j^n$ 项。离散格式为
\begin{equation}
    U_j^{n+1} = \frac 12(U_{j-1}^n + U_{j+1}^n) - \frac {\mu}2(U_{j+1}^n - U_{j-1}^n).
\end{equation}
同样是直接法。这里 $\mu = \dfrac{ak}h$。

\subsubsection{Lax-Wendroff 方法}

在 $u(x,t)$ 处多展开一阶,离散格式为
\begin{equation}
    U_j^{n+1} = U_j^n - \frac {\mu}2(U_{j+1}^n - U_{j-1}^n) + \frac{\mu^2}2(U_{j+1}^n - 2U_j^n + U_{j-1}^n).
\end{equation}
同样是直接法。

\subsubsection{Upwind 方法}

注意到当 $a>0$ 时 $U_j^n$ 的值仅依赖其左侧的点,$a<0$ 反之。故离散格式为
\begin{equation}
    U^{n+1}_j = \left\{
    \begin{aligned}
        U^n_j - \mu(U^n_j - U^n_{j-1}), & a\geq 0\\
        U^n_j - \mu(U^n_{j+1} - U^n_j), & a < 0.
    \end{aligned}
    \right.
\end{equation}
同样是直接法。

\subsubsection{Beam-Warming 方法}

在 Upwind 方法的基础上再多展开一阶,就得到了 Beam-Warming 方法。
\begin{equation}
    U^{n+1}_j = \left\{
    \begin{aligned}
        U^n_j - \frac{\mu}2(3U^n_j - 4U^n_{j-1} + U^n_{j-2}) + \frac{\mu^2}2(U^n_j - 2U_{j-1}^n + U_{j-2}^n), & a\geq 0\\
        U^n_j - \frac{\mu}2(-3U^n_j + 4U^n_{j+1} - U^n_{j+2}) + \frac{\mu^2}2(U^n_j - 2U_{j+1}^n + U_{j+2}^n), & a < 0. \\
    \end{aligned}
    \right.
\end{equation}
同样是直接法。

\newpage

\section{数据测试}

\subsection{热方程的测试}

取 $g_0=0, g_1=0, f(x) = \max\left(0, 1+2\bigg|x-\dfrac 12\bigg|\right)$。

该热方程没有解析解,但可以用 Fourier 变换计算出级数形式的精确解
\begin{equation}
    u(t,x) = \sum_{k=1}^\infty \frac {40}{k^2\pi^2}\left[-\sin\left(\dfrac{9}{20}k\pi\right) + 2\sin\left(\dfrac 12k\pi\right) - \sin\left(\dfrac{11}{20}k\pi\right)\right]e^{-k^2\pi^2t}\sin(k\pi x).
\end{equation}
取级数前 10 项即可控制误差在 $10^{-20}$ 以内,可以作为“真解”。

\subsubsection{测试1:FTCS方法}

作出FTCS方法在 $r=\dfrac 12, r=1$ 时 $t=k,2k,10k$ 的图像。

\begin{figure}[H]
    \centering
    \begin{subfigure}[b]{0.30\textwidth}
        \includegraphics[width=\textwidth]{../res/FTCS(r=0.5,t=1).png}
        \caption{$r=0.5, t=k$}
    \end{subfigure}
    \hfill
    \begin{subfigure}[b]{0.30\textwidth}
        \includegraphics[width=\textwidth]{../res/FTCS(r=0.5,t=2).png}
        \caption{$r=0.5, t=2k$}
    \end{subfigure}
    \hfill
    \begin{subfigure}[b]{0.30\textwidth}
        \includegraphics[width=\textwidth]{../res/FTCS(r=0.5,t=10).png}
        \caption{$r=0.5, t=10k$}
    \end{subfigure}
%   \end{figure}
%   \begin{figure}[H]
%     \centering
    \begin{subfigure}[b]{0.30\textwidth}
        \includegraphics[width=\textwidth]{../res/FTCS(r=1,t=1).png}
        \caption{$r=1, t=k$}
    \end{subfigure}
    \hfill
    \begin{subfigure}[b]{0.30\textwidth}
        \includegraphics[width=\textwidth]{../res/FTCS(r=1,t=2).png}
        \caption{$r=1, t=2k$}
    \end{subfigure}
    \hfill
    \begin{subfigure}[b]{0.30\textwidth}
        \includegraphics[width=\textwidth]{../res/FTCS(r=1,t=10).png}
        \caption{$r=1, t=10k$}
    \end{subfigure}
  \end{figure}

该方法存在严重震荡,这也证实了它的无条件不稳定性。

\subsubsection{测试2:Crank-Nicolson方法}

作出Crank-Nicolson方法在 $r=1, r=2$ 时 $t=k,2k,10k$ 的图像。

\begin{figure}[H]
    \centering
    \begin{subfigure}[b]{0.30\textwidth}
        \includegraphics[width=\textwidth]{../res/CN(r=1,t=1).png}
        \caption{$r=1, t=k$}
    \end{subfigure}
    \hfill
    \begin{subfigure}[b]{0.30\textwidth}
        \includegraphics[width=\textwidth]{../res/CN(r=1,t=2).png}
        \caption{$r=1, t=2k$}
    \end{subfigure}
    \hfill
    \begin{subfigure}[b]{0.30\textwidth}
        \includegraphics[width=\textwidth]{../res/CN(r=1,t=10).png}
        \caption{$r=1, t=10k$}
    \end{subfigure}
%   \end{figure}
%   \begin{figure}[H]
    % \centering
    \begin{subfigure}[b]{0.30\textwidth}
        \includegraphics[width=\textwidth]{../res/CN(r=2,t=1).png}
        \caption{$r=2, t=k$}
    \end{subfigure}
    \hfill
    \begin{subfigure}[b]{0.30\textwidth}
        \includegraphics[width=\textwidth]{../res/CN(r=2,t=2).png}
        \caption{$r=2, t=2k$}
    \end{subfigure}
    \hfill
    \begin{subfigure}[b]{0.30\textwidth}
        \includegraphics[width=\textwidth]{../res/CN(r=2,t=10).png}
        \caption{$r=2, t=10k$}
    \end{subfigure}
  \end{figure}

方法在 $N=10$ 时已经基本收敛,且 $r=1$ 时收敛速度更快。

\subsubsection{测试3:BTCS方法}

作出BTCS方法在 $r=1, r=10$ 时 $t=k,2k,10k$ 的图像。

\begin{figure}[H]
    \centering
    \begin{subfigure}[b]{0.30\textwidth}
        \includegraphics[width=\textwidth]{../res/BTCS(r=1,t=1).png}
        \caption{$r=1, t=k$}
    \end{subfigure}
    \hfill
    \begin{subfigure}[b]{0.30\textwidth}
        \includegraphics[width=\textwidth]{../res/BTCS(r=1,t=2).png}
        \caption{$r=1, t=2k$}
    \end{subfigure}
    \hfill
    \begin{subfigure}[b]{0.30\textwidth}
        \includegraphics[width=\textwidth]{../res/BTCS(r=1,t=10).png}
        \caption{$r=1, t=10k$}
    \end{subfigure}
%   \end{figure}
%   \begin{figure}[H]
%     \centering
    \begin{subfigure}[b]{0.30\textwidth}
        \includegraphics[width=\textwidth]{../res/BTCS(r=10,t=1).png}
        \caption{$r=10, t=k$}
    \end{subfigure}
    \hfill
    \begin{subfigure}[b]{0.30\textwidth}
        \includegraphics[width=\textwidth]{../res/BTCS(r=10,t=2).png}
        \caption{$r=10, t=2k$}
    \end{subfigure}
    \hfill
    \begin{subfigure}[b]{0.30\textwidth}
        \includegraphics[width=\textwidth]{../res/BTCS(r=10,t=10).png}
        \caption{$r=10, t=10k$}
    \end{subfigure}
  \end{figure}

$r=1,N=10$ 时已经基本收敛;$r=2,N=10$ 时算法还有明显误差。

\subsubsection{测试4:Collocation RK方法的测试}

作出Collocation方法在 $r=1, r=\dfrac 1{2h}$ 时 $t=k,2k,10k$ 的图像。

\begin{figure}[H]
    \centering
    \begin{subfigure}[b]{0.30\textwidth}
        \includegraphics[width=\textwidth]{../res/RKM2(r=1,t=1).png}
        \caption{$r=1, t=k$}
    \end{subfigure}
    \hfill
    \begin{subfigure}[b]{0.30\textwidth}
        \includegraphics[width=\textwidth]{../res/RKM2(r=1,t=2).png}
        \caption{$r=1, t=2k$}
    \end{subfigure}
    \hfill
    \begin{subfigure}[b]{0.30\textwidth}
        \includegraphics[width=\textwidth]{../res/RKM2(r=1,t=10).png}
        \caption{$r=1, t=10k$}
    \end{subfigure}
%   \end{figure}
%   \begin{figure}[H]
%     \centering
    \begin{subfigure}[b]{0.30\textwidth}
        \includegraphics[width=\textwidth]{../res/RKM2(r=10,t=1).png}
        \caption{$r=10, t=k$}
    \end{subfigure}
    \hfill
    \begin{subfigure}[b]{0.30\textwidth}
        \includegraphics[width=\textwidth]{../res/RKM2(r=10,t=2).png}
        \caption{$r=10, t=2k$}
    \end{subfigure}
    \hfill
    \begin{subfigure}[b]{0.30\textwidth}
        \includegraphics[width=\textwidth]{../res/RKM2(r=10,t=10).png}
        \caption{$r=10, t=10k$}
    \end{subfigure}
  \end{figure}

由于其极高的收敛阶,它在 $N=2$ 时就基本收敛。

\subsubsection{测试5:一阶Gauss-Legendre方法的测试}

作出Gauss-Legendre方法在 $r=1, r=\dfrac 1{2h}$ 时 $t=k,2k,10k$ 的图像。

\begin{figure}[H]
    \centering
    \begin{subfigure}[b]{0.30\textwidth}
        \includegraphics[width=\textwidth]{../res/GLRKM1(r=1,t=1).png}
        \caption{$r=1, t=k$}
    \end{subfigure}
    \hfill
    \begin{subfigure}[b]{0.30\textwidth}
        \includegraphics[width=\textwidth]{../res/GLRKM1(r=1,t=2).png}
        \caption{$r=1, t=2k$}
    \end{subfigure}
    \hfill
    \begin{subfigure}[b]{0.30\textwidth}
        \includegraphics[width=\textwidth]{../res/GLRKM1(r=1,t=10).png}
        \caption{$r=1, t=10k$}
    \end{subfigure}
%   \end{figure}
%   \begin{figure}[H]
%     \centering
    \begin{subfigure}[b]{0.30\textwidth}
        \includegraphics[width=\textwidth]{../res/GLRKM1(r=10,t=1).png}
        \caption{$r=10, t=k$}
    \end{subfigure}
    \hfill
    \begin{subfigure}[b]{0.30\textwidth}
        \includegraphics[width=\textwidth]{../res/GLRKM1(r=10,t=2).png}
        \caption{$r=10, t=2k$}
    \end{subfigure}
    \hfill
    \begin{subfigure}[b]{0.30\textwidth}
        \includegraphics[width=\textwidth]{../res/GLRKM1(r=10,t=10).png}
        \caption{$r=10, t=10k$}
    \end{subfigure}
  \end{figure}

$r=1$ 时,这个方法和Crank-Nicolson方法实际上是一种方法,因此图像也完全相同。
但 $r=\dfrac 1{2h}$ 时,Gauss-Legendre方法出现了明显的振荡。

\subsection{平流方程的测试}

取 $a=1, f(x) = \exp\{-20(x-2)^2\}+\exp\{-(x-5)^2\}, \Omega = [15,25]\times (0,17)$,则
\begin{equation}
    u(t,x) = f(x-t) = \exp\{-20(x-t-2)^2\} + \exp\{-(x-t-5)^2\}.
\end{equation}

我们作出 $t=T$ 时的以下图像:
\begin{enumerate}
    \item 真解
    \item leapfrog($k=0.8h$)
    \item upwind($k=0.8h$)
    \item Beam-Warming($k=0.8h$)
    \item Lax-Friedrichs($k=0.8h$)
    \item Lax-Wendroff($k=0.8h$)
    \item Lax-Wendroff($k=h$)
    \item leapfrog($k=h$)
\end{enumerate}

\begin{figure}[H]
    \centering
    \begin{subfigure}[b]{0.30\textwidth}
        \includegraphics[width=\textwidth]{../res/leapfrog(k=0.8h).png}
        \caption{leapfrog,$k=0.8h$}
    \end{subfigure}
    \hfill
    \begin{subfigure}[b]{0.30\textwidth}
        \includegraphics[width=\textwidth]{../res/upwind(k=0.8h).png}
        \caption{upwind,$k=0.8h$}
    \end{subfigure}
    \hfill
    \begin{subfigure}[b]{0.30\textwidth}
        \includegraphics[width=\textwidth]{../res/BW(k=0.8h).png}
        \caption{Beam-Warming,$k=0.8h$}
    \end{subfigure}
%   \end{figure}
%   \begin{figure}[H]
    % \centering
    \begin{subfigure}[b]{0.23\textwidth}
        \includegraphics[width=\textwidth]{../res/LF(k=0.8h).png}
        \caption{Lax-Friedrichs,$k=0.8h$}
    \end{subfigure}
    \hfill
    \begin{subfigure}[b]{0.23\textwidth}
        \includegraphics[width=\textwidth]{../res/LW(k=0.8h).png}
        \caption{Lax-Wendroff,$k=0.8h$}
    \end{subfigure}
    \hfill
    \begin{subfigure}[b]{0.23\textwidth}
        \includegraphics[width=\textwidth]{../res/LW(k=h).png}
        \caption{Lax-Wendroff,$k=h$}
    \end{subfigure}
    \hfill
    \begin{subfigure}[b]{0.23\textwidth}
        \includegraphics[width=\textwidth]{../res/leapfrog(k=h).png}
        \caption{leapfrog,$k=h$}
    \end{subfigure}
  \end{figure}

在 $k=0.8h$ 时,各种方法都有较大误差,
upwind和Lax-Friedrichs主要体现在峰的大小上;
而其他算法则体现在峰值附近的振荡和峰值位置的偏差。
而在 $k=h$ 时,这里展示的两种方法求出的解都是精确的,
事实上所有五种方法都应该完全精确,因为从依赖域上看,它们在初值上的依赖域都是单点,即真解的点。

\end{document}